%!TEX root = forallxyyc.tex
\part{Natural deduction for TFL}
\label{ch.NDTFL}
\addtocontents{toc}{\protect\mbox{}\protect\hrulefill\par}

\chapter{The very idea of natural deduction}\label{s:NDVeryIdea}

Way back in \cref{s:Valid}, we said that an argument is valid \ifeff{} there is no case in which all of the premises are true and the conclusion is false.

In the case of TFL, this led us to develop truth tables. Each line of a complete truth table corresponds to a valuation. So, when faced with a TFL argument, we have a very direct way to assess whether there is a valuation on which the premises are true and the conclusion is false: just thrash through the truth table.

However, truth tables may not give us much \emph{insight}. Consider two arguments in TFL:
\begin{align*}
P \eor Q, \enot P & \therefore Q\\
P \eif Q, P & \therefore Q
\end{align*}
Clearly, these are valid arguments. You can confirm that they are valid by constructing four-line truth tables, but we might say that they make use of different \emph{forms} of reasoning. It might be nice to keep track of these different forms of inference.

One aim of a \emph{natural deduction system} is to show that particular arguments are valid, in a way that allows us to understand the reasoning that the arguments might involve. We begin with very basic rules of inference. These rules can be combined to offer more complicated arguments. Indeed, with just a small starter pack of rules of inference, we hope to capture all valid arguments.

\emph{This is a very different way of thinking about arguments.} 

With truth tables, we directly consider different ways to make sentences true or false. With natural deduction systems, we manipulate sentences in accordance with rules that we have set down as good rules. The latter promises to give us a better insight---or at least, a different insight---into how arguments work.

The move to natural deduction might be motivated by more than the search for insight. It might also be motivated by \emph{necessity}. Consider:
\ifHTMLtarget
\begin{earg}
	\item $A_1 \eif C_1$
	\item[\texttherefore] $(A_1 \eand (A_2 \eand (A_3 \eand (A_4 \eand A_5)))) \eif 
	((((C_1 \eor C_2) \eor C_3) \eor C_4) \eor C_5)$
\end{earg}
\else
\begin{align*}
	& A_1 \eif C_1 \\
	\therefore\, & (A_1 \eand (A_2 \eand (A_3 \eand (A_4 \eand A_5)))) \eif {}\\ 
	& \qquad ((((C_1 \eor C_2) \eor C_3) \eor C_4) \eor C_5)
\end{align*}
\fi
To test this argument for validity, you might use a 1{,}024-line truth table. If you do it correctly, then you will see that there is no line on which all the premises are true and on which the conclusion is false. So you will know that the argument is valid. (But, as just mentioned, there is a sense in which you will not know \emph{why} the argument is valid.) But now consider:
\ifHTMLtarget
\begin{earg}
\item $A_1 \eif C_1$
\item[\texttherefore] $(A_1 \eand (A_2 \eand (A_3 \eand (A_4 \eand (A_5 \eand 
 (A_6 \eand (A_7 \eand (A_8 \eand (A_9 \eand A_{10}))))))))) \eif 
 (((((((((C_1 \eor C_2) \eor C_3) \eor C_4) \eor C_5) \eor  C_6) \eor
 C_7) \eor C_8) \eor C_9) \eor C_{10})$
\end{earg}
\else
\begin{align*}
& A_1 \eif C_1 \\
\therefore\, & \small (A_1 \eand (A_2 \eand (A_3 \eand (A_4 \eand (A_5 \eand {}\\
&\qquad (A_6 \eand (A_7 \eand (A_8 \eand (A_9 \eand A_{10}))))))))) \eif {}\\
&(((((((((C_1 \eor C_2) \eor C_3) \eor C_4) \eor C_5) \eor {}\\
& \qquad C_6) \eor C_7) \eor C_8) \eor C_9) \eor C_{10})
\end{align*}
\fi
This argument is also valid---as you can probably tell---but to test it requires a truth table with $2^{20} = 1{,}048{,}576$ lines. In principle, we can set a machine to grind through truth tables and report back when it is finished. In practice, complicated arguments in TFL can become \emph{intractable} if we use truth tables.

When we get to first-order logic (FOL) (beginning in \cref{s:FOLBuildingBlocks}), though, the problem gets dramatically worse. There is nothing like the truth table test for FOL. To assess whether or not an argument is valid, we have to reason about \emph{all} interpretations, but, as we will see, there are infinitely many possible interpretations. We cannot even in principle set a machine to grind through infinitely many possible interpretations and report back when it is finished: it will \emph{never} finish. We either need to come up with some more efficient way of reasoning about all interpretations, or we need to look for something different.

There are, indeed, systems that codify ways to reason about all possible interpretations. They were developed in the 1950s by Evert Beth and Jaakko Hintikka, but we will not follow this path. We will, instead, look to natural deduction.

Rather than reasoning directly about all valuations (in the case of TFL), we will try to select a few basic rules of inference. Some of these will govern the behaviour of the sentential connectives. Others will govern the behaviour of the quantifiers and identity that are the hallmarks of FOL. The resulting system of rules will give us a new way to think about the validity of arguments.
The modern development of natural deduction dates from simultaneous and unrelated papers by Gerhard Gentzen and Stanis\l{}aw Ja\'{s}kowski (1934). However, the natural deduction system that we will consider is based largely around work by Frederic Fitch (first published in 1952).



\chapter{Basic rules for TFL}\label{s:BasicTFL}
We will develop a \define{natural deduction} system. For each connective, there will be \define{introduction} rules, that allow us to prove a sentence that has that connective as the main logical operator, and \define{elimination} rules, that allow us to prove something given a sentence that has that connective as the main logical operator.

\section{The idea of a formal proof}
A \emph{formal proof} or \emph{derivation} is a sequence of sentences,
some of which are marked as being initial assumptions (or premises).
The last line of the formal proof is the conclusion. (Henceforth, we
will simply call these `proofs' or `derivations', but be aware that
there are \emph{informal proofs} too.)

As an illustration, consider:
	$$\enot (A \eor B) \therefore \enot A \eand \enot B$$
We will start a proof by writing the premise:
\begin{fitchproof}
	\hypo{a1}{\enot (A \eor B)}\PR
\end{fitchproof}
Note that we have numbered the premise, since we will want to refer back to it. Indeed, every line of the proof is numbered, so that we can refer back to it.

Note also that we have drawn a line underneath the premise. Everything written above the line is an \emph{assumption}. Everything written below the line will either be something which follows from the assumptions, or it will be some new assumption. We are hoping to conclude `$\enot A \eand \enot B$'; so we are hoping ultimately to conclude our proof with
\begin{fitchproof}
	\have[n]{con}{\enot A \eand \enot B}
\end{fitchproof}
for some number $n$. It doesn't matter what line number we end on, but we would obviously prefer a short proof to a long one.

Similarly, suppose we wanted to consider:
$$A\eor B, \enot (A\eand C), \enot (B \eand \enot D) \therefore \enot C\eor D$$
The argument has three premises, so we start by writing them all down, numbered, and drawing a line under them:
\begin{fitchproof}
	\hypo{a1}{A \eor B}\PR
	\hypo{a2}{\enot (A\eand C)}\PR
	\hypo{a3}{\enot (B \eand \enot D)}\PR
\end{fitchproof}
and we are hoping to conclude with some line:
\begin{fitchproof}
	\have[n]{con}{\enot C \eor D}
\end{fitchproof}
All that remains to do is to explain each of the rules that we can use along the way from premises to conclusion. The rules are broken down by our logical connectives.

\section{Reiteration}
The very first rule is so breathtakingly obvious that it is surprising we bother with it at all.

If you already have shown something in the course of a proof, the \emph{reiteration rule} allows you to repeat it on a new line. For example:
\begin{fitchproof}
	\have[4]{a1}{A \eand B}
	\ellipsesline
	\have[10]{a2}{A \eand B} \by{R}{a1}
\end{fitchproof}
This indicates that we have written `$A \eand B$' on line~$4$. Now, at some later line---line~$10$, for example---we have decided that we want to repeat this. So we write it down again. We also add a citation which justifies what we have written. In this case, we write `R', to indicate that we are using the reiteration rule, and we write `$4$', to indicate that we have applied it to line $4$.

Here is a general expression of the rule:
\factoidbox{
\begin{fitchproof}
	\have[m]{a}{\metav{A}}
	\have[\ ]{c}{\metav{A}} \by{R}{a}
\end{fitchproof}}
The point is that, if any sentence $\metav{A}$ occurs on some line, then we can repeat $\metav{A}$ on later lines. Each line of our proof must be justified by some rule, and here we have `R $m$'. This means: Reiteration, applied to line~$m$.

Two things need emphasizing. First `$\metav{A}$' is not a sentence of TFL. Rather, it a symbol in the metalanguage, which we use when we want to talk about any sentence of TFL (see \cref{s:UseMention}). Second, and similarly, `$m$' is not a symbol that will appear on a proof. Rather, it is a symbol in the metalanguage, which we use when we want to talk about any line number of a proof. In an actual proof, the lines are numbered `$1$', `$2$', `$3$', and so forth. But when we define the rule, we use variables like `$m$' to underscore the point that the rule may be applied at any point.

\section{Conjunction}
Suppose we want to show that Ludwig is both reactionary and libertarian. One obvious way to do this would be as follows: first we show that Ludwig is reactionary; then we show that Ludwig is libertarian; then we put these two demonstrations together, to obtain the conjunction.

Our natural deduction system will capture this thought straightforwardly. In the example given, we might adopt the following symbolization key:
	\begin{ekey}
		\item[R] Ludwig is reactionary
		\item[L] Ludwig is libertarian
	\end{ekey}
Perhaps we are working through a proof, and we have obtained `$R$' on
line~$8$ and `$L$' on line~$15$. Then on any subsequent line we can
obtain `$R \eand L$' thus:
\begin{fitchproof}
	\have[8]{a}{R}
	\have[15]{b}{L}
	\have[\ ]{c}{R \eand L} \ai{a, b}
\end{fitchproof}
Note that every line of our proof must either be an assumption, or
must be justified by some rule. We cite `$\eand$I $8$, $15$' here to
indicate that the line is obtained by the rule of conjunction
introduction ($\eand$I) applied to lines $8$ and~$15$. We could
equally well obtain:
\begin{fitchproof}
	\have[8]{a}{R}
	\have[15]{b}{L}
	\have[\ ]{c}{L \eand R} \ai{b, a}
\end{fitchproof}
with the citation reversed, to reflect the order of the conjuncts. More generally, here is our conjunction introduction rule:
\factoidbox{
\begin{fitchproof}
	\have[m]{a}{\metav{A}}
	\have[n]{b}{\metav{B}}
	\have[\ ]{c}{\metav{A}\eand\metav{B}} \ai{a, b}
\end{fitchproof}}
To be clear, the statement of the rule is \emph{schematic}. It is not itself a proof.  `$\metav{A}$' and `$\metav{B}$' are not sentences of TFL. Rather, they are symbols in the metalanguage, which we use when we want to talk about any sentence of TFL (see \cref{s:UseMention}). Similarly, `$m$' and `$n$' are not numerals that will appear in any actual proof. Rather, they are symbols in the metalanguage, which we use when we want to talk about any line number of any proof. In an actual proof, the lines are numbered `$1$', `$2$', `$3$', and so forth, but when we define the rule, we use variables to emphasize that the rule may be applied at any point. The rule requires only that we have both conjuncts available to us somewhere earlier in the proof. They can be separated from one another, and they can appear in any order.

The rule is called `conjunction \emph{introduction}' because it introduces the symbol `$\eand$' into our proof where it may have been absent. Correspondingly, we have a rule that \emph{eliminates} that symbol.  Suppose you have shown that Ludwig is both reactionary and libertarian. You are entitled to conclude that Ludwig is reactionary. Equally, you are entitled to conclude that Ludwig is libertarian. Putting this together, we obtain our conjunction elimination rule(s):
\factoidbox{
\begin{fitchproof}
	\have[m]{ab}{\metav{A}\eand\metav{B}}
	\have[\ ]{a}{\metav{A}} \ae{ab}
\end{fitchproof}}
and equally:
\factoidbox{
\begin{fitchproof}
	\have[m]{ab}{\metav{A}\eand\metav{B}}
	\have[\ ]{b}{\metav{B}} \ae{ab}
\end{fitchproof}}
The point is simply that, when you have a conjunction on some line of a proof, you can obtain either of the conjuncts by~{\eand}E. One point is worth emphasising: you can only apply this rule when conjunction is the main logical operator. So you cannot infer `$D$' just from `$C \eor (D \eand E)$'!

Even with just these two rules, we can start to see some of the power of our formal proof system. Consider:
\begin{earg}
\item[] $[(A\eor B)\eif(C\eor D)] \eand [(E \eor F) \eif (G\eor H)]$
\item[\texttherefore] $[(E \eor F) \eif (G\eor H)] \eand [(A\eor B)\eif(C\eor D)]$
\end{earg}
The main logical operator in both the premise and conclusion of this argument is `$\eand$'. In order to provide a proof, we begin by writing down the premise, which is our assumption. We draw a line below this: everything after this line must follow from our assumptions by (repeated applications of) our rules of inference. So the beginning of the proof looks like this:
\begin{fitchproof}
	\hypo{ab}{{[}(A\eor B)\eif(C\eor D){]} \eand {[}(E \eor F) \eif (G\eor H){]}}\PR
\end{fitchproof}
From the premise, we can get each of the conjuncts by {\eand}E. The proof now looks like this:
\begin{fitchproof}
	\hypo{ab}{{[}(A\eor B)\eif(C\eor D){]} \eand {[}(E \eor F) \eif (G\eor H){]}}\PR
	\have{a}{{[}(A\eor B)\eif(C\eor D){]}} \ae{ab}
	\have{b}{{[}(E \eor F) \eif (G\eor H){]}} \ae{ab}
\end{fitchproof}
So by applying the {\eand}I rule to lines 3 and 2 (in that order), we arrive at the desired conclusion. The finished proof looks like this:
\begin{fitchproof}
	\hypo{ab}{{[}(A\eor B)\eif(C\eor D){]} \eand {[}(E \eor F) \eif (G\eor H){]}}\PR

	\have{a}{{[}(A\eor B)\eif(C\eor D){]}} \ae{ab}
	\have{b}{{[}(E \eor F) \eif (G\eor H){]}} \ae{ab}
	\have{ba}{{[}(E \eor F) \eif (G\eor H){]} \eand {[}(A\eor B)\eif(C\eor D){]}} \ai{b,a}
\end{fitchproof}
This is a very simple proof, but it shows how we can chain rules of proof together into longer proofs. In passing, note that investigating this argument with a truth table would have required 256 lines; our formal proof required only four lines.

It is worth giving another example. Back in \cref{s:MoreBracketingConventions}, we noted that this argument is valid:
	$$A \eand (B \eand C) \therefore (A \eand B) \eand C$$
To provide a proof corresponding to this argument, we start by writing:
\begin{fitchproof}
	\hypo{ab}{A \eand (B \eand C)}\PR
\end{fitchproof}
From the premise, we can get each of the conjuncts by applying~$\eand$E twice. We can then apply $\eand$E twice more, so our proof looks like:
\begin{fitchproof}
	\hypo{ab}{A \eand (B \eand C)}\PR
	\have{a}{A} \ae{ab}
	\have{bc}{B \eand C} \ae{ab}
	\have{b}{B} \ae{bc}
	\have{c}{C} \ae{bc}
\end{fitchproof}
But now we can merrily reintroduce conjunctions in the order we wanted them, so that our final proof is:
\begin{fitchproof}
	\hypo{abc}{A \eand (B \eand C)}\PR
	\have{a}{A} \ae{abc}
	\have{bc}{B \eand C} \ae{abc}
	\have{b}{B} \ae{bc}
	\have{c}{C} \ae{bc}
	\have{ab}{A \eand B}\ai{a, b}
	\have{con}{(A \eand B) \eand C}\ai{ab, c}
\end{fitchproof}
Recall that our official definition of sentences in TFL only allowed conjunctions with two conjuncts. The proof just given suggests that we could drop inner brackets in all of our proofs. However, this is not standard, and we will not do this. Instead, we will maintain our more austere bracketing conventions. (Though we will still allow ourselves to drop outermost brackets, for legibility.)

Let's give one final illustration. When using the $\eand$I rule, there is no requirement to apply it to different sentences. So, if we want, we can formally prove `$A \eand A$' from `$A$' thus:
\begin{fitchproof}
	\hypo{a}{A}\PR
	\have{aa}{A \eand A}\ai{a, a}
\end{fitchproof}
Simple, but effective.

\section{Conditional}
Consider the following argument:
\begin{earg}
		\item[] If Jane is smart then she is fast.
		\item[] Jane is smart.
		\item[\texttherefore] Jane is fast.
\end{earg}
This argument is certainly valid, and it suggests a straightforward conditional elimination rule ($\eif$E):
\factoidbox{
\begin{fitchproof}
	\have[m]{ab}{\metav{A}\eif\metav{B}}
	\have[n]{a}{\metav{A}}
	\have[\ ]{b}{\metav{B}} \ce{ab,a}
\end{fitchproof}}
This rule is also sometimes called \emph{modus ponens}. Again, this is an elimination rule, because it allows us to obtain a sentence that may not contain `$\eif$', having started with a sentence that did contain `$\eif$'. Note that the conditional $\metav{A}\eif\metav{B}$ and the antecedent~$\metav{A}$ can be separated from one another in the proof, and they can appear in any order. However, in the citation for $\eif$E, we always cite the conditional first, followed by the antecedent.

The rule for conditional introduction is also quite easy to motivate. The following argument should be valid:
	\begin{earg}
		\item Ludwig is reactionary. 
		\item[\texttherefore] If Ludwig is libertarian, then Ludwig is both reactionary \emph{and} libertarian.
	\end{earg}
If someone doubted that this was valid, we might try to convince them otherwise by explaining ourselves as follows:
	\begin{quote}
		Assume that Ludwig is reactionary. Now, \emph{additionally} assume that Ludwig is libertarian. Then by conjunction introduction---which we just discussed---Ludwig is both reactionary and libertarian. Of course, that's conditional on the assumption that Ludwig is libertarian. But this just means that, if Ludwig is libertarian, then he is both reactionary and libertarian.
	\end{quote}
Transferred into natural deduction format, here is the pattern of reasoning that we just used. We started with one premise, `Ludwig is reactionary', thus:
	\begin{fitchproof}
		\hypo{r}{R}\PR
	\end{fitchproof}
The next thing we did is to make an \emph{additional} assumption (`Ludwig is libertarian'), for the sake of argument. To indicate that we are no longer dealing \emph{merely} with our original premise (`$R$'), but with some additional assumption, we continue our proof as follows:
	\begin{fitchproof}
		\hypo{r}{R}\PR
		\open
			\hypo{l}{L}\AS
	\end{fitchproof}
Note that we are \emph{not} claiming, on line~$2$, to have proved
`$L$' from line~$1$, so we do not write in any justification for the
additional assumption on line~$2$. We do, however, need to mark that
it is an additional assumption. We do this by drawing a line under it
(to indicate that it is an assumption) and by indenting it with a
further vertical line (to indicate that it is additional).

With this extra assumption in place, we are in a position to
use~$\eand$I. So we can continue our proof:
	\begin{fitchproof}
		\hypo{r}{R}\PR
		\open
			\hypo{l}{L}\AS
			\have{rl}{R \eand L}\ai{r, l}
%			\close
%		\have{con}{L \eif (R \eand L)}\ci{l-rl}
	\end{fitchproof}
So we have now shown that, on the additional assumption, `$L$', we can obtain `$R \eand L$'. We can therefore conclude that, if `$L$' obtains, then so does `$R \eand L$'. Or, to put it more briefly, we can conclude `$L \eif (R \eand L)$':
	\begin{fitchproof}
		\hypo{r}{R}\PR
		\open
			\hypo{l}{L}\AS
			\have{rl}{R \eand L}\ai{r, l}
			\close
		\have{con}{L \eif (R \eand L)}\ci{l-rl}
	\end{fitchproof}
Observe that we have dropped back to using one vertical line on the left.  We have \emph{discharged} the additional assumption, `$L$', since the conditional itself follows just from our original assumption,~`$R$'.

The general pattern at work here is the following. We first make an additional assumption, $\metav{A}$; and from that additional assumption, we prove~$\metav{B}$. In that case, we know the following: If~$\metav{A}$ is true, then~$\metav{B}$ is true. This is wrapped up in the rule for conditional introduction:
\factoidbox{
	\begin{fitchproof}
		\open
			\hypo[i]{a}{\metav{A}}\AS
			\have[j]{b}{\metav{B}}
		\close
		\have[\ ]{ab}{\metav{A}\eif\metav{B}}\ci{a-b}
	\end{fitchproof}}
There can be as many or as few lines as you like between lines $i$ and $j$.

It will help to offer a second  illustration of $\eif$I in action. Suppose we want to consider the following:
	$$P \eif Q, Q \eif R \therefore P \eif R$$
We start by listing \emph{both} of our premises. Then, since we want to arrive at a conditional (namely, `$P \eif R$'), we additionally assume the antecedent to that conditional. Thus our main proof starts:
\begin{fitchproof}
	\hypo{pq}{P \eif Q}\PR
	\hypo{qr}{Q \eif R}\PR
	\open
		\hypo{p}{P}\AS
	\close
\end{fitchproof}
Note that we have made `$P$' available, by treating it as an additional assumption, but now, we can use {\eif}E on the first premise. This will yield `$Q$'. We can then use {\eif}E on the second premise. So, by assuming `$P$' we were able to prove `$R$', so we apply the {\eif}I rule---discharging `$P$'---and finish the proof. Putting all this together, we have:
\label{HSproof}
\begin{fitchproof}
	\hypo{pq}{P \eif Q}\PR
	\hypo{qr}{Q \eif R}\PR
	\open
		\hypo{p}{P}\AS
		\have{q}{Q}\ce{pq,p}
		\have{r}{R}\ce{qr,q}
	\close
	\have{pr}{P \eif R}\ci{p-r}
\end{fitchproof}


\section{Additional assumptions and subproofs}
The rule $\eif$I invoked the idea of making additional assumptions. These need to be handled with some care. Consider this proof:
\begin{fitchproof}
	\hypo{a}{A}\PR
	\open
		\hypo{b1}{B}\AS
		\have{b2}{B} \by{R}{b1}
	\close
	\have{con}{B \eif B}\ci{b1-b2}
\end{fitchproof}
This is perfectly in keeping with the rules we have laid down already, and it should not seem particularly strange. Since `$B \eif B$' is a tautology, no particular premises should be required to prove it.

But suppose we now tried to continue the proof as follows:
\begin{fitchproof}
	\hypo{a}{A}\PR
	\open
		\hypo{b1}{B}\AS
		\have{b2}{B} \by{R}{b1}
	\close
	\have{con}{B \eif B}\ci{b1-b2}
\ifHTMLtarget
	\have{b}{B} \by{naughty attempt to invoke $\eif$E}{con, b2}
\else
	\have{b}{B} \by{naughty attempt}{}
	\have [\ ]{x}{} \by{to invoke $\eif$E}{con, b2}
\fi
\end{fitchproof}
If we were allowed to do this, it would be a disaster. It would allow us to prove any sentence letter from any other sentence letter. However, if you tell me that Anne is fast (symbolized by `$A$'), we shouldn't be able to conclude that Queen Boudica stood twenty-feet tall (symbolized by `$B$')! We must be prohibited from doing this, but how are we to implement the prohibition?

We can describe the process of making an additional assumption as one of performing a \emph{subproof}: a subsidiary proof within the main proof. When we start a subproof, we draw another vertical line to indicate that we are no longer in the main proof. Then we write in the assumption upon which the subproof will be based. A subproof can be thought of as essentially posing this question: \emph{what could we show, if we also make this additional assumption?}

When we are working within the subproof, we can refer to the additional assumption that we made in introducing the subproof, and to anything that we obtained from our original assumptions. (After all, those original assumptions are still in effect.) At some point though, we will want to stop working with the additional assumption: we will want to return from the subproof to the main proof. To indicate that we have returned to the main proof, the vertical line for the subproof comes to an end. At this point, we say that the subproof is \define{closed}. Having closed a subproof, we have set aside the additional assumption, so it will be illegitimate to draw upon anything that depends upon that additional assumption. Thus we stipulate:
\factoidbox{To cite an individual line when applying a rule:
\begin{compactlist}
\item the line must come before the line where the rule is applied, but 
\item not occur within a subproof that has been closed before the line where the rule is applied.
\end{compactlist}}
This stipulation rules out the disastrous attempted proof above. The application of rule $\eif$E on line~$5$ requires that we cite two individual lines from earlier in the proof. One of these lines (namely, line~$3$) occurs within a subproof (lines $2$--$3$). By line $5$, where we have to cite line $3$, the subproof has been closed. This is illegitimate: we are not allowed to cite line $3$ on line $5$.

Closing a subproof is called \define{discharging} the assumptions of that subproof. So we can put the point this way: \emph{you cannot refer back to anything that was obtained using discharged assumptions}.

Subproofs, then, allow us to think about what we could show, if we made additional assumptions. The point to take away from this is not surprising---in the course of a proof, we have to keep very careful track of what assumptions we are making, at any given moment. Our proof system does this very graphically. (Indeed, that's precisely why we have chosen to use \emph{this} proof system.)

Once we have started thinking about what we can show by making additional assumptions, nothing stops us from posing the question of what we could show if we were to make \emph{even more} assumptions? This might motivate us to introduce a subproof within a subproof. Here is an example using only the rules which we have considered so far: 
\begin{fitchproof}
\hypo{a}{A}\PR
\open
	\hypo{b}{B}\AS
	\open
		\hypo{c}{C}\AS
		\have{ab}{A \eand B}\ai{a,b}
	\close
	\have{cab}{C \eif (A \eand B)}\ci{c-ab}
\close
\have{bcab}{B \eif (C \eif (A \eand B))}\ci{b-cab}
\end{fitchproof}
Notice that the citation on line~$4$ refers back to the initial
assumption (on line~$1$) and an assumption of a subproof (on
line~$2$). This is perfectly in order, since neither assumption has
been discharged at the time (i.e., by line~$4$).

Again, though, we need to keep careful track of what we are assuming at any given moment. Suppose we tried to continue the proof as follows:
\begin{fitchproof}
\hypo{a}{A}\PR
\open
	\hypo{b}{B}\AS
	\open
		\hypo{c}{C}\AS
		\have{ab}{A \eand B}\ai{a,b}
	\close
	\have{cab}{C \eif (A \eand B)}\ci{c-ab}
\close
\have{bcab}{B \eif(C \eif (A \eand B))}\ci{b-cab}
\ifHTMLtarget
	\have{bcab}{C \eif (A \eand B)}\by{naughty attempt to invoke $\eif$I}{c-ab}
\else
	\have{bcab}{C \eif (A \eand B)}\by{naughty attempt}{}
	\have [\ ]{x}{} \by{to invoke $\eif$I}{c-ab}
\fi
\end{fitchproof}
This would be awful. If we tell you that Anne is smart, you should not
be able to infer that, if Cath is smart (symbolized by `$C$') then
\emph{both} Anne is smart and Queen Boudica stood 20~feet tall! But
this is just what such a proof would suggest, if it were permissible.

The essential problem is that the subproof that began with the assumption~`$C$' depended crucially on the fact that we had assumed `$B$' on line~$2$. By line~$6$, we have \emph{discharged} the assumption~`$B$': we have stopped asking ourselves what we could show, if we also assumed `$B$'. So it is simply cheating, to try to help ourselves (on line~$7$) to the subproof that began with the assumption~`$C$'. Thus we stipulate, much as before, that a subproof can only be cited on a line if it does not occur within some other subproof which is already closed at that line. The attempted disastrous proof violates this stipulation. The subproof of lines $3$--$4$ occurs within a subproof that ends on line~$5$. So it cannot be invoked on line~$7$.

Here is one further case we have to exclude:
\begin{fitchproof}
\hypo{a}{A}\PR
\open
	\hypo{b}{B}\AS
	\open
	\hypo{c}{C}\AS
	\have{bc}{B \eand C}\ai{b,c}
	\have{c2}{C}\ae{bc}
	\close
\close
\ifHTMLtarget
\have{bcab}{B \eif C}\by{naughty attempt to invoke $\eif$I}{b-c2}
\else
\have{bcab}{B \eif C}\by{naughty attempt}{}
\have [\ ]{x}{} \by{to invoke $\eif$I}{b-c2}
\fi
\end{fitchproof}
Here we are trying to cite a subproof that begins on line~$2$ and ends on line~$5$---but the sentence on line~$5$ depends not only on the assumption on line~$2$, but also on one another assumption (line~$3$) which we have not discharged at the end of the subproof. The subproof started on line~$3$ is still open at line~$3$. But $\eif$I requires that the last line of the subproof \emph{only} relies on the assumption of the subproof being cited, i.e., the subproof beginning on line~$2$ (and anything before it), and not on assumptions of any subproofs within it. In particular, the last line of the subproof cited must not itself lie within a nested subproof.

\factoidbox{To cite a subproof when applying a rule:
\begin{compactlist} 
\item the cited subproof must come entirely before the application of the rule where it is cited, 
\item the cited subproof must not lie within some other closed subproof which is closed at the line it is cited, and 
\item the last line of the cited subproof must not occur inside a nested subproof.
\end{compactlist}}

One last point to emphasize how rules can be applied: where a rule requires you to cite an individual line, you cannot cite a subproof instead; and where it requires you to cite a subproof, you cannot cite an individual line instead. So for instance, this is incorrect:
\begin{fitchproof}
\hypo{a}{A}\PR
\open
	\hypo{b}{B}\AS
	\open
	\hypo{c}{C}\AS
	\have{bc}{B \eand C}\ai{b,c}
	\have{c2}{C}\ae{bc}
	\close
\ifHTMLtarget
	\have{c3}{C}\by{naughty attempt to invoke R}{c-c2}
\else
\have{c3}{C}\by{naughty attempt}{}
\have [\ ]{x}{} \by{to invoke R}{c-c2}
\fi
\close
\have[7]{bcab}{B \eif C}\ci{b-c3}
\end{fitchproof}
Here, we have tried to justify $C$ on line~$6$ by the reiteration rule, but we have cited the subproof on lines $3$--$5$ with it. That subproof is closed and can in principle be cited on line~$6$. (For instance, we could use it to justify $C \eif C$ by $\eif$I.) But the reiteration rule~R requires you to cite an individual line, so citing the entire subproof is inadmissible (even if that subproof contains the sentence~$C$ we want to reiterate).


It is always permissible to open a subproof with any assumption. However, there is some strategy involved in picking a useful assumption. Starting a subproof with an arbitrary, wacky assumption would just waste lines of the proof. In order to obtain a conditional by {\eif}I, for instance, you must assume the antecedent of the conditional in a subproof.

Equally, it is always permissible to close a subproof (and discharge
its assumptions). However, it will not be helpful to do so until you
have reached something useful. Once the subproof is closed, you can
only cite the entire subproof in any justification. Those rules that
call for a subproof or subproofs, in turn, require that the last line
of the subproof is a sentence of some form or other. For instance, you
are only allowed to cite a subproof for~$\eif$I if the line you are
justifying is of the form $\metav{A} \eif \metav{B}$, $\metav{A}$ is
the assumption of your subproof, and $\metav{B}$ is the last line of
your subproof.


\section{Biconditional}

The rules for the biconditional will be like double-barrelled versions
of the rules for the conditional.

In order to prove `$F \eiff G$', for instance, you must be able to
prove `$G$' on the assumption `$F$' \emph{and} prove `$F$' on the
assumption~`$G$'. The biconditional introduction rule ({\eiff}I)
therefore requires two subproofs. Schematically, the rule works like
this: \factoidbox{
\begin{fitchproof}
	\open
		\hypo[i]{a1}{\metav{A}}\AS
		\have[j]{b1}{\metav{B}}
	\close
	\open
		\hypo[k]{b2}{\metav{B}}\AS
		\have[l]{a2}{\metav{A}}
	\close
	\have[\ ]{ab}{\metav{A}\eiff\metav{B}}\bi{a1-b1,b2-a2}
\end{fitchproof}}
There can be as many lines as you like between $i$ and $j$, and as many lines as you like between $k$ and $l$. Moreover, the subproofs can come in any order, and the second subproof does not need to come immediately after the first.

The biconditional elimination rule ({\eiff}E) lets you do a bit more than the conditional rule. If you have the left-hand subsentence of the biconditional, you can obtain the right-hand subsentence. If you have the right-hand subsentence, you can obtain the left-hand subsentence. So we allow:
\factoidbox{
\begin{fitchproof}
	\have[m]{ab}{\metav{A}\eiff\metav{B}}
	\have[n]{a}{\metav{A}}
	\have[\ ]{b}{\metav{B}} \be{ab,a}
\end{fitchproof}}
and equally:
\factoidbox{\begin{fitchproof}
	\have[m]{ab}{\metav{A}\eiff\metav{B}}
	\have[n]{a}{\metav{B}}
	\have[\ ]{b}{\metav{A}} \be{ab,a}
\end{fitchproof}}
Note that the biconditional, and the right or left half, can be separated from one another, and they can appear in any order. However, in the citation for $\eiff$E, we always cite the biconditional first.

\section{Disjunction}
Suppose Ludwig is reactionary. Then Ludwig is either reactionary or libertarian. After all, to say that Ludwig is either reactionary or libertarian is to say something weaker than to say that Ludwig is reactionary.

Let's emphasize this point. Suppose Ludwig is reactionary. It follows that Ludwig is \emph{either} reactionary \emph{or} a kumquat. Equally, it follows that \emph{either} Ludwig is reactionary \emph{or} kumquats are the only fruit.  Equally, it follows that \emph{either} Ludwig is reactionary \emph{or}  God is dead. Many of these are strange inferences to draw, but there is nothing \emph{logically} wrong with them (even if they maybe violate all sorts of implicit conversational norms).

Armed with all this, we present the disjunction introduction rule(s):
\factoidbox{\begin{fitchproof}
	\have[m]{a}{\metav{A}}
	\have[\ ]{ab}{\metav{A}\eor\metav{B}}\oi{a}
\end{fitchproof}}
and
\factoidbox{\begin{fitchproof}
	\have[m]{a}{\metav{A}}
	\have[\ ]{ba}{\metav{B}\eor\metav{A}}\oi{a}
\end{fitchproof}}
Notice that \metav{B} can be \emph{any} sentence whatsoever, so the following is a perfectly acceptable proof:
\begin{fitchproof}
	\hypo{m}{M}\PR
	\have{mmm}{M \eor ([(A\eiff B) \eif (C \eand D)] \eiff [E \eand F])}\oi{m}
\end{fitchproof}
Using a truth table to show this would have taken 128 lines.

The disjunction elimination rule is slightly trickier. Suppose that either Ludwig is reactionary or he is libertarian. What can you conclude? Not that Ludwig is reactionary; it might be that he is libertarian instead. Equally, not that Ludwig is libertarian; for he might merely be reactionary. Disjunctions, just by themselves, are hard to work with.

But suppose that we could somehow show both of the following: first, that Ludwig's being reactionary entails that he is an Austrian economist; second, that Ludwig's being libertarian entails that he is an Austrian economist. Then if we know that Ludwig is either reactionary or libertarian, then we know that, whichever he is, Ludwig is an Austrian economist. This insight can be expressed in the following rule, which is our disjunction elimination ($\eor$E) rule:
\factoidbox{
	\begin{fitchproof}
		\have[m]{ab}{\metav{A}\eor\metav{B}}
		\open
			\hypo[i]{a}{\metav{A}} \AS
			\have[j]{c1}{\metav{C}}
		\close
		\open
			\hypo[k]{b}{\metav{B}} \AS
			\have[l]{c2}{\metav{C}}
		\close
		\have[ ]{c}{\metav{C}}\oe{ab, a-c1,b-c2}
	\end{fitchproof}}
This is obviously a bit clunkier to write down than our previous rules, but the point is fairly simple. Suppose we have some disjunction, $\metav{A} \eor \metav{B}$. Suppose we have two subproofs, showing us that $\metav{C}$ follows from the assumption that $\metav{A}$, and that $\metav{C}$ follows from the assumption that $\metav{B}$. Then we can infer $\metav{C}$ itself. As usual, there can be as many lines as you like between $i$ and $j$, and as many lines as you like between $k$ and $l$. Moreover, the subproofs and the disjunction can come in any order, and do not have to be adjacent.

Some examples might help illustrate this. Consider this argument:
$$(P \eand Q) \eor (P \eand R) \therefore P$$
An example proof might run thus:
	\begin{fitchproof}
		\hypo{prem}{(P \eand Q) \eor (P \eand R) }\PR
			\open
				\hypo{pq}{P \eand Q}\AS
				\have{p1}{P}\ae{pq}
			\close
			\open
				\hypo{pr}{P \eand R}\AS
				\have{p2}{P}\ae{pr}
			\close
		\have{con}{P}\oe{prem, pq-p1, pr-p2}
	\end{fitchproof}
Here is a slightly harder example. Consider:
	$$ A \eand (B \eor C) \therefore (A \eand B) \eor (A \eand C)$$
Here is a proof corresponding to this argument:
	\begin{fitchproof}
		\hypo{aboc}{A \eand (B \eor C)}\PR
		\have{a}{A}\ae{aboc}
		\have{boc}{B \eor C}\ae{aboc}
		\open
			\hypo{b}{B}\AS
			\have{ab}{A \eand B}\ai{a,b}
			\have{abo}{(A \eand B) \eor (A \eand C)}\oi{ab}
		\close
		\open
			\hypo{c}{C}\AS
			\have{ac}{A \eand C}\ai{a,c}
			\have{aco}{(A \eand B) \eor (A \eand C)}\oi{ac}
		\close
	\have{con}{(A \eand B) \eor (A \eand C)}\oe{boc, b-abo, c-aco}
	\end{fitchproof}
Don't be alarmed if you think that you wouldn't have been able to come up with this proof yourself. The ability to come up with novel proofs comes with practice, and we'll cover some strategies for finding proofs in \cref{s:stratTFL}. The key question at this stage is whether, looking at the proof, you can see that it conforms to the rules that we have laid down. That just involves checking every line, and making sure that it is justified in accordance with the rules we have laid down.


\section{Contradiction and negation}

We have only one connective left to deal with: negation. But to tackle it, we must connect negation with \emph{contradiction}.

An effective form of argument is to argue your opponent into contradicting themselves. At that point, you have them on the ropes. They have to give up at least one of their assumptions. We are going to make use of this idea in our proof system, by adding a new symbol, `$\ered$', to our proofs. This should be read as something like `contradiction!'\ or `reductio!'\ or `but that's absurd!'  The rule for introducing this symbol is that we can use it whenever we explicitly contradict ourselves, i.e., whenever we find both a sentence and its negation appearing in our proof:
\factoidbox{
\begin{fitchproof}
  \have[m]{na}{\enot\metav{A}}
  \have[n]{a}{\metav{A}}
  \have[ ]{bot}{\ered}\ne{na, a}
\end{fitchproof}}
It does not matter what order the sentence and its negation appear in, and they do not need to appear on adjacent lines. However, we always cite the line number of the negation first, followed by that of the sentence it is a negation of.

There is obviously a tight link between contradiction and negation.
The rule $\enot$E lets us proceed from two contradictory sentences---$\metav{A}$ and its negation $\enot \metav{A}$---to an explicit contradition~$\ered$. We choose the label for a reason: it is the the most basic rule that lets us proceed from a premise containing a negation, i.e., $\enot\metav{A}$, to a sentence not containing it, i.e., $\ered$. So it is a rule that \emph{eliminates}~$\enot$.

We have said that `$\ered$' should be read as something like `contradiction!' but this does not tell us much about the symbol. There are, roughly, three ways to approach the symbol.
	\begin{compactlist}
		\item We might regard `$\ered$' as a new atomic sentence of TFL, but one which can only ever have the truth value False.
		\item We might regard `$\ered$' as an abbreviation for some canonical contradiction, such as `$A \eand \enot A$'. This will have the same effect as the above---obviously, `$A \eand \enot A$' only ever has the truth value False---but it means that, officially, we do not need to add a new symbol to TFL.
		\item We might regard `$\ered$', not as a symbol of TFL, but as something more like a \emph{punctuation mark} that appears in our proofs. (It is on a par with the line numbers and the vertical lines, say.)
	\end{compactlist}
There is something very philosophically attractive about the third option, but here we will \emph{officially} adopt the first. `$\ered$' is to be read as a sentence letter that is always false. This means that we can manipulate it, in our proofs, just like any other sentence.

We still have to state a rule for negation introduction. The rule is very simple: if assuming something leads you to a contradiction, then the assumption must be wrong. This thought motivates the following rule:
\factoidbox{\begin{fitchproof}
\open
	\hypo[i]{a}{\metav{A}}\AS
	\have[j]{nb}{\ered}
\close
\have[\ ]{na}{\enot\metav{A}}\ni{a-nb}
\end{fitchproof}}
There can be as many lines between $i$ and $j$ as you like. To see this in practice, and interacting with negation, consider this proof:
	\begin{fitchproof}
		\hypo{d}{D}\AS
		\open
			\hypo{nd}{\enot D}\AS
			\have{ndr}{\ered}\ne{nd, d}
		\close
		\have{con}{\enot\enot D}\ni{nd-ndr}
	\end{fitchproof}

If the assumption that $\metav{A}$ is true leads to a contradiction, $\metav{A}$ cannot be true, i.e., it must be false, i.e., $\enot\metav{A}$ must be true. Of course, if the assumption that $\metav{A}$ is false (i.e., the assumption that $\enot\metav{A}$ is true) leads to a contradiction, then $\metav{A}$ cannot be false, i.e., $\metav{A}$ must be true. So we can consider the following rule:
\factoidbox{\begin{fitchproof}
\open
	\hypo[i]{a}{\enot\metav{A}}\AS
	\have[j]{nb}{\ered}
\close
\have[\ ]{na}{\metav{A}}\ip{a-nb}
\end{fitchproof}}
This rule is called \emph{indirect proof}, since it allows us to prove $\metav{A}$ indirectly, by assuming its negation. Formally, the rule is very similar to $\enot$I, but $\metav{A}$ and $\enot\metav{A}$ have changed places. Since $\enot\metav{A}$ is not the conclusion of the rule, we are not introducing~$\enot$, so IP is not a rule that introduces any connective. It also doesn't eliminate a connective, since it has no free-standing premises which contain~$\enot$, only a subproof with an assumption of the form~$\enot\metav{A}$. By contrast, $\enot$E does have a premise of the form $\enot\metav{A}$: that's why $\enot$E eliminates~$\enot$, but IP does not.\footnote{There are logicians who have qualms about IP, but not about $\enot$E. They are called ``intuitionists.'' Intuitionists don't buy our basic assumption that every sentence has one of two truth values, true or false. They also think that $\enot$ works differently---for them, a proof of $\ered$ from $\metav{A}$ guarantees $\enot \metav{A}$, but a proof of $\ered$ from $\enot\metav{A}$ does not guarantee that~$\metav{A}$, but only $\enot\enot\metav{A}$. So, for them, $\metav{A}$ and $\enot\enot\metav{A}$ are not equivalent.}

Using $\enot$I, we were able to give a proof of $\enot\enot\metav{D}$ from $\metav{D}$. Using IP, we can go the other direction (with essentially the same proof).
	\begin{fitchproof}
		\hypo{d}{\enot\enot D}\PR
		\open
			\hypo{nd}{\enot D}\AS
			\have{ndr}{\ered}\ne{d, nd}
		\close
		\have{con}{D}\ip{nd-ndr}
	\end{fitchproof}

We need one last rule. It is a kind of elimination rule for `$\ered$', and known as \emph{explosion}.\footnote{The Latin name for this principle is \emph{ex contradictione quodlibet}, ``from contradiction, anything.''} If we obtain a contradiction, symbolized by `$\ered$', then we can infer whatever we like. How can this be motivated, as a rule of argumentation? Well, consider the English rhetorical device `\ldots and if \emph{that's} true, I'll eat my hat'. Since contradictions simply cannot be true, if one \emph{is} true then not only will I eat my hat, I'll have it too. Here is the formal rule:
\factoidbox{\begin{fitchproof}
\have[m]{bot}{\ered}
\have[ ]{}{\metav{A}}\re{bot}
\end{fitchproof}}
Note that \metav{A} can be \emph{any} sentence whatsoever.

The explosion rule is a bit odd. It looks like \metav{A} arrives in our proof like a bunny out of a hat. When trying to find proofs, it is very tempting to try to use it everywhere, since it seems so powerful. Resist this temptation: you can only apply it when you already have~$\ered$!  And you get $\ered$ only when your assumptions are contradictory.

Still, isn't it odd that from a contradiction anything whatsoever should follow? Not according to our notion of entailment and validity. For \metav{A} entails \metav{B} \ifeff{} there is no valuation of the sentence letters which makes \metav{A} true and \metav{B} false at the same time. Now $\ered$ is a contradiction---it is never true, whatever the valuation of the sentence letters. Since there is no valuation which makes $\ered$ true, there of course is also no valuation that makes $\ered$ true and \metav{B} false! So according to our definition of entailment, $\ered \entails \metav{B}$, whatever \metav{B} is. A contradiction entails anything.\footnote{There are some logicians who don't buy this. They think that if \metav{A} entails \metav{B}, there must be some \emph{relevant connection} between \metav{A} and \metav{B}---and there isn't one between $\ered$ and some arbitrary sentence~\metav{B}. So these logicians develop other, ``relevant'' logics in which you aren't allowed the explosion rule.}

\emph{These are all of the basic rules for the proof system for TFL.}

\practiceproblems

\problempart
The following two `proofs' are \emph{incorrect}. Explain the mistakes they make.
\begin{fitchproof}
\hypo{abc}{(\enot L \eand A) \eor L}\PR
\open
\hypo{nla}{\enot L \eand A}\AS
\have{nl}{\enot L}\ae{nl}
	\have{a}{A}\ae{abc}
\close
\open
	\hypo{l}{L}\AS
	\have{red}{\ered}\ne{nl, l}
	\have{a2}{A}\re{red}
\close
\have{con}{A}\oe{abc, nla-a, l-a2}
\end{fitchproof}

\begin{fitchproof}
\hypo{abc}{A \eand (B \eand C)}\PR
\hypo{bcd}{(B \eor C) \eif D}\PR
\have{b}{B}\ae{abc}
\have{bc}{B \eor C}\oi{b}
\have{d}{D}\ce{bc, bcd}
\end{fitchproof}

\problempart
The following three proofs are missing their citations (rule and line numbers). Add them, to turn them into \emph{bona fide} proofs. Additionally, write down the argument that corresponds to each proof.
\begin{multicols}{2}
\begin{fitchproof}
\hypo{ps}{P \eand S}
\hypo{nsor}{S \eif R}
\have{p}{P}%\ae{ps}
\have{s}{S}%\ae{ps}
\have{r}{R}%\ce{nsor, s}
\have{re}{R \eor E}%\oi{r}
\end{fitchproof}

\begin{fitchproof}
\hypo{ad}{A \eif D}
\open
	\hypo{ab}{A \eand B}
	\have{a}{A}%\ae{ab}
	\have{d}{D}%\ce{ad, a}
	\have{de}{D \eor E}%\oi{d}
\close
\have{conc}{(A \eand B) \eif (D \eor E)}%\ci{ab-de}
\end{fitchproof}

\begin{fitchproof}
\hypo{nlcjol}{\enot L \eif (J \eor L)}
\hypo{nl}{\enot L}
\have{jol}{J \eor L}%\ce{nlcjol, nl}
\open
	\hypo{j}{J}
	\have{jj}{J \eand J}%\ai{j}
	\have{j2}{J}%\ae{jj}
\close
\open
	\hypo{l}{L}
	\have{red}{\ered}%\ne{nl, l}
	\have{j3}{J}%\re{red}
\close
\have{conc}{J}%\oe{jol, j-j2, l-j3}
\end{fitchproof}
\end{multicols}

\solutions
\problempart
\label{pr.solvedTFLproofs}
Give a proof for each of the following arguments:
\begin{compactlist}
\item $J\eif\enot J \therefore \enot J$
\item $Q\eif(Q\eand\enot Q) \therefore \enot Q$
\item $A\eif (B\eif C) \therefore (A\eand B)\eif C$
\item $K\eand L \therefore K\eiff L$
\item $(C\eand D)\eor E \therefore E\eor D$
\item $A\eiff B, B\eiff C \therefore A\eiff C$
\item $\enot F\eif G, F\eif H \therefore G\eor H$
\item $(Z\eand K) \eor (K\eand M), K \eif D \therefore D$
\item $P \eand (Q\eor R), P\eif \enot R \therefore Q\eor E$
\item $S\eiff T \therefore S\eiff (T\eor S)$
\item $\enot (P \eif Q) \therefore \enot Q$
\item $\enot (P \eif Q) \therefore P$
\end{compactlist}


\chapter{Constructing proofs}\label{s:stratTFL}

There is no simple recipe for finding proofs, and there is no substitute for practice. Here, though, are some rules of thumb and strategies to keep in mind.

\section{Working backward from what we want}

So you're trying to find a proof of some conclusion~$\metav{C}$, which will be the last line of your proof. The first thing you do is look at~$\metav{C}$ and ask what the introduction rule is for its main logical operator. This gives you an idea of what should happen \emph{before} the last line of the proof. The justifications for the introduction rule require one or two other sentences above the last line, or one or two subproofs. Moreover, you can tell from~$\metav{C}$ what those sentences are, or what the assumptions and conclusions of the subproof(s) are. Then you can write down those sentence or outline the subproof(s) above the last line, and treat those as your new goals.

For example: If your conclusion is a conditional $\metav{A}\eif\metav{B}$, plan to use the {\eif}I rule. This requires starting a subproof in which you assume~\metav{A}. The subproof ought to end with~\metav{B}. Then, continue by thinking about what you should do to get $\metav{B}$ inside that subproof, and how you can use the assumption~$\metav{A}$.

If your goal is a conjunction, conditional, or negated sentence, you should start by working backward in this way. We'll describe what you have to do in each of these cases in detail.

\subsection*{Working backward from a conjunction}

If we want to prove $\metav{A} \eand \metav{B}$, working backward means we should write $\metav{A} \eand \metav{B}$ at the bottom of our proof, and try to prove it using $\eand$I. At the top, we'll write out the premises of the proof, if there are any. Then, at the bottom, we write the sentence we want to prove. If it is a conjunction, we'll prove it using $\eand$I.
  \begin{fitchproof}
	\hypo{1}{\metav{P}_1}\PR
	\ellipsesline 
	\hypo[k]{k}{\metav{P}_k}\PR
\ellipsesline
    \have[n]{n}{\metav{A}}{} 
    \ellipsesline 
	\have[m]{m}{\metav{B}}
    \have{4}{\metav{A} \eand \metav{B}}\ai{n,m}
  \end{fitchproof}
For $\eand$I, we need to prove $\metav{A}$ first, then prove
$\metav{B}$. For the last line, we have to cite the lines where we
(will have) proved $\metav{A}$ and  $\metav{B}$, and use~$\eand$I. The
parts of the proof labelled by the vertical $\cdots$ have to still be filled in.
We'll mark the line numbers $m$, $n$ for now. When the proof is
complete, these placeholders can be replaced by actual numbers.

\subsection*{Working backward from a conditional}

If our goal is to prove a conditional $\metav{A} \eif \metav{B}$, we'll have to use $\eif$I. This requires a subproof starting with $\metav{A}$ and ending with~$\metav{B}$. We'll set up our proof as follows:
\begin{fitchproof}
\open
\hypo[n]{2}{\metav{A}}
\ellipsesline 
\have[m]{3}{\metav{B}}
\close
\have{4}{\metav{A} \eif \metav{B}}\ci{2-3}
\end{fitchproof} 
Again we'll leave placeholders in the line number slots. We'll record the last inference as $\eif$I, citing the subproof.

\subsection*{Working backward from a negated sentence}

If we want to prove $\enot \metav{A}$, we'll have to use $\enot$I.
\begin{fitchproof}
\open
\hypo[n]{2}{\metav{A}}
\ellipsesline 
\have[m]{3}{\ered}
\close
\have{4}{\enot \metav{A}}\ni{2-3}
\end{fitchproof} 
For $\enot$I, we have to start a subproof with assumption $\metav{A}$; the last line of the subproof has to be $\ered$.  We'll cite the subproof, and use~$\enot$I as the rule.

When working backward, continue to do so as long as you can. So if you're working backward to prove $\metav{A} \eif \metav{B}$ and have set up a subproof in which you want to prove $\metav{B}$. Now look at~$\metav{B}$. If, say, it is a conjunction, work backward from it, and write down the two conjuncts inside your subproof. Etc.

\subsection*{Working backward from a disjunction}

Of course, you can also work backward from a disjunction $\metav{A} \eor \metav{B}$, if that is your goal.
The $\eor$I rule requires that you have one of the disjuncts in order to infer $\metav{A} \eor \metav{B}$.
So to work backward, you pick a disjunct, infer $\metav{A} \eor \metav{B}$ from it, and then continue to look for a proof of the disjunct you picked:
\begin{fitchproof}
	\ellipsesline
	\have[n]{2}{\metav{A}} 
	\have{3}{\metav{A} \eor \metav{B}}\oi{2}
\end{fitchproof}
However, you may not be able to prove the disjunct you picked. In that
case you have to backtrack. When you can't fill in the part labelled
by the vertical $\cdots$,
delete everything, and try with the other disjunct:
\begin{fitchproof}
	\ellipsesline 
	\have[n]{2}{\metav{B}} 
	\have{3}{\metav{A} \eor \metav{B}}\oi{2}
\end{fitchproof}
Obviously, deleting everything and starting over is frustrating, so you should avoid it. If your goal is a disjunction, therefore, you should \emph{not start} by working backward: try working forward first, and apply the $\eor$I strategy only when working forward (and working backward using $\eand$I, $\eif$I, and $\enot$I) no longer work.

\section{Work forward from what you have}

Your proof may have premises. And if you've worked backward in order to prove a conditional or a negated sentence, you will have set up subproofs with an assumption, and be looking to prove a final sentence in the subproof. These premises and assumptions are sentences you can work forward from in order to fill in the missing steps in your proof. That means applying elimination rules for the main operators of these sentences. The form of the rules will tell you what you'll have to do.

\subsection*{Working forward from a conjunction}

To work forward from a sentence of the form $\metav{A} \eand \metav{B}$, we use $\eand$E. That rule allows us to do two things: infer $\metav{A}$, and infer $\metav{B}$. So in a proof where we have $\metav{A} \eand \metav{B}$, we can work forward by writing $\metav{A}$ and/or $\metav{B}$ immediately below the conjunction:
\begin{fitchproof}
  \have[n]{1}{\metav{A} \eand \metav{B}}
  \have{2}{\metav{A}}\ae{1}
  \have{3}{\metav{B}}\ae{1}
\end{fitchproof}
Usually it will be clear in the particular situation you're in which one of \metav{A} or \metav{B} you'll need. It doesn't hurt, however, to write them both down.

\subsection*{Working forward from a disjunction}

Working forward from a disjunction works a bit differently. To use a disjunction, we use the $\eor$E rule. In order to apply that rule, it is not enough to know what the disjuncts of the disjunction are that we want to use. We must also keep in mind what we want to prove. Suppose we want to prove~$\metav{C}$, and we have $\metav{A} \eor B$ to work with. (That $\metav{A} \eor B$ may be a premise of the proof, an assumption of a subproof, or something already proved.) In order to be able to apply the $\eor$E rule, we'll have to set up two subproofs:
\begin{fitchproof}
	\have[n]{1}{\metav{A} \eor \metav{B}}
	\open
	\hypo{2}{\metav{A}} 
	\ellipsesline 
	\have[m]{3}{\metav{C}}
	\close 
	\open
	\hypo{4}{\metav{B}}
	\ellipsesline
	\have[k]{5}{\metav{C}}
	\close
	\have{6}{\metav{C}}\oe{1,(2)-3,(4)-5} 
\end{fitchproof} 
The first subproof starts with the first disjunct, $\metav{A}$, and ends with the sentence we're looking for, $\metav{C}$. The second subproof starts with the other disjunct, $\metav{B}$, and also ends with the goal sentence~$\metav{C}$. Each of these subproofs have to be filled in further. We can then justify the goal sentence $\metav{C}$ by using $\eor$E, citing the line with $\metav{A} \eor \metav{B}$ and the two subproofs.

\subsection*{Working forward from a conditional}

In order to use a conditional $\metav{A} \eif \metav{B}$, you also need the antecedent $\metav{A}$ in order to apply~$\eif$E. So to work forward from a conditional, you will derive $\metav{B}$, justify it by $\eif$E, and set up $\metav{A}$ as a new subgoal.
\begin{fitchproof}
	\have[n]{1}{\metav{A} \eif \metav{B}}
	\ellipsesline 
	\have[m]{2}{\metav{A}}
	\have{3}{\metav{B}}\ce{1,2} 
\end{fitchproof}

\subsection*{Working forward from a negated sentence}

Finally, to use a negated sentence $\enot \metav{A}$, you would apply $\enot$E. It requires, in addition to $\enot \metav{A}$, also the corresponding sentence~$\metav{A}$ without the negation. The sentence you'll get is always the same: $\ered$. So working forward from a negated sentence works especially well inside a subproof that you'll want to use for $\enot$I (or IP). You work forward from $\enot \metav{A}$ if you already have $\enot \metav{A}$ and you want to prove~$\ered$. To do it, you set up $\metav{A}$ as a new subgoal.
\begin{fitchproof}
	\have[n]{1}{\enot \metav{A}}
	\ellipsesline 
	\have[m]{2}{\metav{A}}
	\have{3}{\ered}\ne{1,2} 
\end{fitchproof}

\section{Strategies at work}

Suppose we want to show that the argument $(A \eand B) \eor (A \eand C) \therefore A \eand (B \eor C)$ is valid. We start the proof by writing the premise and conclusion down. (On a piece of paper, you would want as much space as possible between them, so write the premises at the top of the sheet and the conclusion at the bottom.)
\begin{fitchproof}
   \hypo{1}{(A \eand B) \eor (A \eand C)}\PR
\ellipsesline
  \have[n]{2}{A \eand (B \eor C)}
\end{fitchproof}
We now have two options: either work backward from the conclusion, or work forward from the premise. We'll pick the second strategy: we use the disjunction on line~$1$, and set up the subproofs we need for $\eor$E. The disjunction on line~$1$ has two disjuncts, $A \eand B$ and $A \eand C$. The goal sentence you want to prove is $A \eand (B \eor C)$. So in this case you have to set up two subproofs, one with assumption $A \eand B$ and last line $A \eand (B \eor C)$, the other with assumption $A \eand C$ and last line $A \eand (B \eor C)$. The justification for the conclusion on line~$n$ will be $\eor$E, citing the disjunction on line~$1$ and the two subproofs. So your proof now looks like this:
\begin{fitchproof}
	\hypo{1}{(A \eand B) \eor (A \eand C)}\PR
	\open
	\hypo{2}{A \eand B}\AS
	\ellipsesline 
	\have[n]{6}{A \eand (B \eor C)}
	\close
	\open
	\hypo{7}{A \eand C}\AS
	\ellipsesline
	\have[m]{11}{A \eand (B \eor C)}
	\close
	\have{12}{A \eand (B \eor C)}\oe{1,2-6,(7)-11}
\end{fitchproof}
You now have two separate tasks, namely to fill in each of the two subproofs. In the first subproof, we now work backward from the conclusion $A \eand (B \eor C)$. That is a conjunction, so inside the first subproof, you will have two separate subgoals: proving $A$, and proving $B \eor C$. These subgoals will let you justify line~$n$ using~$\eand$I. Your proof now looks like this:
\begin{fitchproof}
	\hypo{1}{(A \eand B) \eor (A \eand C)}\PR
	\open
	\hypo{2}{A \eand B}\AS
	\ellipsesline
	\have[i]{4}{A}
	\ellipsesline
	\have[n][-1]{5}{B \eor C}
	\have[n]{6}{A \eand (B \eor C)}\ai{4,5}
	\close
	\open
	\hypo{7}{A \eand C}\AS
	\ellipsesline
	\have[m]{11}{A \eand (B \eor C)}
	\close
	\have{12}{A \eand (B \eor C)}\oe{1,2-6,(7)-11}
\end{fitchproof}
We immediately see that we can get line $i$ from line~$2$ by $\eand$E. So line~$i$ is actually line~$3$, and can be justified with $\eand$E from line~$2$. The other subgoal $B \eor C$ is a disjunction. We'll apply the strategy for working backward from a disjunctions to line $n-1$. We have a choice of which disjunct to pick as a subgoal, $B$ or~$C$. Picking $C$ wouldn't work and we'd end up having to backtrack. And you can already see that if you pick $B$ as a subgoal, you could get that by working forward again from the conjunction $A \eand B$ on line~$2$. So we can complete the first subproof as follows:
\begin{fitchproof}
	\hypo{1}{(A \eand B) \eor (A \eand C)}\PR
	\open
	\hypo{2}{A \eand B}\AS
	\have{3}{A}\ae{2}
	\have{4}{B}\ae{2}
	\have{5}{B \eor C}\oi{4}
	\have{6}{A \eand (B \eor C)}\ai{3,5}
	\close
	\open
	\hypo{7}{A \eand C}\AS
	\ellipsesline
	\have[m]{11}{A \eand (B \eor C)}
	\close
	\have{12}{A \eand (B \eor C)}\oe{1,2-6,7-11}
\end{fitchproof}
Like line~$3$, we get line $4$ from $2$ by $\eand$E. Line~$5$ is justified by $\eor$I from line~$4$, since we were working backward from a disjunction there.

That's it for the first subproof. The second subproof is almost exactly the same. We'll leave it as an exercise.

Remember that when we started, we had the option of working forward from the premise, or working backward from the conclusion, and we picked the first option. The second option also leads to a proof, but it will look different.  The first steps would be to work backward from the conclusion and set up two subgoals, $A$ and $B \eor C$, and then work forward from the premise to prove them, e.g.,
\begin{fitchproof}
	\hypo{1}{(A \eand B) \eor (A \eand C)}\PR
	\open
	\hypo{2}{A \eand B}\AS
	\ellipsesline
	\have[k]{3}{A}
	\close
	\open
	\hypo{4}{A \eand C}\AS
	\ellipsesline
	\have[n][-1]{5}{A}
	\close
	\have{6}{A}\oe{1,2-3,(4)-(5)}
	\open
	\hypo{7}{A \eand B}\AS
	\ellipsesline
	\have[l]{8}{B \eor C}
	\close
	\open
	\hypo{9}{A \eand C}\AS
	\ellipsesline
	\have[m][-1]{10}{B \eor C}
	\close
	\have{11}{B \eor C}\oe{1,(7)-8,(9)-(10)}	
	\have{12}{A \eand (B \eor C)}\ai{6,11}
\end{fitchproof}
We'll leave you to fill in the missing pieces indicated by~$\vdots$.

Let's give another example to illustrate how to apply the strategies to deal with conditionals and negation. The sentence $(A \eif B) \eif (\enot B \eif \enot A)$ is a tautology. Let's see if we can find a proof of it, from no premises, using the strategies. We first write the sentence at the bottom of a sheet of paper. Since working forward is not an option (there is nothing to work forward from), we work backward, and set up a subproof to establish the sentence we want, $(A \eif B) \eif (\enot B \eif \enot A)$, using $\eif$I. Its assumption must be the antecedent of the conditional we want to prove, i.e., $A \eif B$, and its last line the consequent, $\enot B \eif \enot A$.
\begin{fitchproof}
\open
\hypo{1}{A \eif B}\AS
\ellipsesline
\have[n]{7}{\enot B \eif \enot A}
\close
\have{8}{(A \eif B) \eif (\enot B \eif \enot A)}\ci{1-7}
\end{fitchproof}
The new goal, $\enot B \eif \enot A$ is itself a conditional, so working backward we set up another subproof:
\begin{fitchproof}
	\open
	\hypo{1}{A \eif B}\AS
	\open
	\hypo{2}{\enot B}\AS
	\ellipsesline
	\have[n][-1]{6}{\enot A}
	\close
	\have{7}{\enot B \eif \enot A}\ci{2-(6)}
	\close
	\have{8}{(A \eif B) \eif (\enot B \eif \enot A)}\ci{1-7}
\end{fitchproof}
From $\enot A$ we again work backward. To do this, look at the $\enot$I rule. It requires a subproof with~$A$ as assumption, and $\ered$ as its last line. So the proof is now:
\begin{fitchproof}
	\open
	\hypo{1}{A \eif B}\AS
	\open
	\hypo{2}{\enot B}\AS
	\open\hypo{3}{A}
	\ellipsesline
	\have[n][-2]{5}{\ered}
	\close
	\have{6}{\enot A}\ni{3-(5)}
	\close
	\have{7}{\enot B \eif \enot A}\ci{2-(6)}
	\close
	\have{8}{(A \eif B) \eif (\enot B \eif \enot A)}\ci{1-7}
\end{fitchproof}
Now our goal is to prove~$\ered$. We said above, when discussing how
to work forward from a negated sentence, that the $\enot$E rule allows
you to prove~$\ered$, which is our goal in the innermost subproof. So
we look for a negated sentence which we can work forward from: that
would be $\enot B$ on line~$2$. That means we have to derive $B$
inside the subproof, since $\enot$E requires not just $\enot B$ (which
we have already), but also~$B$. And $B$, in turn, we get by working
forward from $A \eif B$, since $\eif$E will allow us to justify the
consequent of that conditional, $B$, by $\eif$E. The rule $\eif$E also
requires the antecedent~$A$ of the conditional, but that is also
already available (on line~$3$). So we finish with:
\begin{fitchproof}
	\open
	\hypo{1}{A \eif B}\AS
	\open
	\hypo{2}{\enot B}\AS
	\open\hypo{3}{A}
	\have{4}{B}\ce{1,3}
	\have{5}{\ered}\ne{2,4}
	\close
	\have{6}{\enot A}\ni{3-5}
	\close
	\have{7}{\enot B \eif \enot A}\ci{2-6}
	\close
	\have{8}{(A \eif B) \eif (\enot B \eif \enot A)}\ci{1-7}
\end{fitchproof}

\section{Working forward from $\ered$}\label{sec:backred}

When applying the strategies, you will sometimes find yourself in a situation where you can justify~$\ered$. Using the explosion rule, this would allow you to justify \emph{anything}. So $\ered$ works like a wildcard in proofs. For instance, suppose you want to give a proof of the argument $A \eor B, \enot A \therefore B$. You set up your proof, writing the premises $A \eor B$ and $\enot A$ at the top on lines $1$ and $2$, and the conclusion~$B$ at the bottom of the page. $B$ has no main connective, so you can't work backward from it. Instead, you must work forward from $A \eor B$: That requires two subproofs, like so:
\begin{fitchproof}
	\hypo{1}{A \eor B}\PR
	\hypo{7}{\enot A}\PR
	\open
	\hypo{2}{A} \AS
	\ellipsesline 
	\have[m]{3}{B}
	\close 
	\open
	\hypo{4}{B}\AS
	\ellipsesline
	\have[k]{5}{B}
	\close
	\have{6}{B}\oe{1,2-3,(4)-5} 
\end{fitchproof} 
Notice that you have $\enot A$ on line~$2$ and $A$ as the assumption of your first subproof. That gives you $\ered$ using $\enot$E, and from $\ered$ you get the conclusion~$B$ of the first subroof using~X. Recall that you can repeat a sentence you already have by using the reiteration rule~R. So our proof would be:
\begin{fitchproof}
	\hypo{1}{A \eor B}\PR
	\hypo{7}{\enot A}\PR
	\open
	\hypo{2}{A} \AS
	\have{8}{\ered}\ne{7,2} 
	\have{3}{B}\re{8}
	\close 
	\open
	\hypo{4}{B}\AS
	\have{5}{B}\by{R}{4}
	\close
	\have{6}{B}\oe{1,2-3,4-5} 
\end{fitchproof} 

\section{Proceed indirectly}

In very many cases, the strategies of working forward and backward will eventually pan out. But there are cases where they do not work.  If you cannot find a way to show $\metav{A}$ directly using those, use IP instead. To do this, set up a subproof in which you assume $\enot\metav{A}$ and look for a proof of $\ered$ inside that subproof.

\begin{fitchproof}
\open
\hypo[n]{2}{\enot A}\AS
\ellipsesline 
\have[m]{3}{\ered}
\close
\have{4}{\metav{A}}\ip{2-3}
\end{fitchproof}
Here, we have to start a subproof with assumption $\enot \metav{A}$;
the last line of the subproof has to be~$\ered$. We'll cite the subproof, and use~IP as the rule.  In the subproof, we now have an additional assumption (on line $n$) to work with.

Suppose we used the indirect proof strategy, or we're in some other situation where we're looking for a proof of $\ered$.  What's a good candidate? Of course the obvious candidate would be to use a negated sentence, since (as we saw above) $\enot$E always yields~$\ered$. If you set up a proof as above, trying to prove \metav{A} using~IP, you will have $\enot \metav{A}$ as the assumption of your subproof---so working forward from it to justify $\ered$ inside your subproof, you would next set up \metav{A} as a goal inside your subproof. If you are using this IP strategy, you will find yourself in the following situation: 
\begin{fitchproof}
\open
\hypo[n]{2}{\enot \metav{A}}\AS
\ellipsesline
\have[m][-1]{3}{\metav{A}}
\have{4}{\ered}\ne{2,3}
\close
\have{5}{\metav{A}}\ip{2-4}
\end{fitchproof} 
This looks weird: We wanted to prove $\metav{A}$ and the strategies failed us; so we used IP as a last resort. And now we find ourselves in the same situation: we are again looking for a proof of~$\metav{A}$. But notice that we are now \emph{inside} a subproof, and in that subproof we have an additional assumption ($\enot \metav{A}$) to work with which we didn't have before. Let's look at an example.

\section{Indirect proof of excluded middle}\label{s:proofLEM}

The sentence $A \eor \enot A$ is a tautology, and so should have a proof even without any premises. But working backward fails us: to get $A \eor \enot A$ using $\eor$I we would have to prove either $A$ or $\enot A$---again, from no premises. Neither of these is a tautology, so we won't be able to prove either. Working forward doesn't work either, since there is nothing to work forward from. So, the only option is indirect proof.
\begin{fitchproof}
	\open
	\hypo{2}{\enot (A \eor \enot A)}\AS
	\ellipsesline
	\have[m]{8}{\ered}
	\close
	\have{9}{A \eor \enot A}\ip{2-8}
\end{fitchproof}
Now we do have something to work forward from: the assumption $\enot(A \eor \enot A)$. To use it, we justify $\ered$ by $\enot$E, citing the assumption on line~$1$, and also the corresponding unnegated sentence $A \eor \enot A$, yet to be proved.
\begin{fitchproof}
	\open
	\hypo{2}{\enot (A \eor \enot A)}\AS
	\ellipsesline
	\have[m][-1]{7}{A \eor \enot A}
	\have{8}{\ered}\ne{2,7}
	\close
	\have{9}{A \eor \enot A}\ip{2-8}
\end{fitchproof}
At the outset, working backward to prove $A \eor\enot A$ by $\eor$I did not work. But we are now in a different situation: we want to prove $A \eor\enot A$ inside a subproof. In general, when dealing with new goals we should go back and start with the basic strategies. In this case, we should first try to work backward from the disjunction $A \eor \enot A$, i.e., we have to pick a disjunct and try to prove it. Let's pick~$\enot A$. This would let us justify $A \eor \enot A$ on line~$m - 1$ using $\eor$I. Then working backward from $\enot A$, we start another subproof in order to justify $\enot A$ using $\enot$I. That subproof must have $A$ as the assumption and~$\ered$ as its last line.
\begin{fitchproof}
	\open
	\hypo{2}{\enot (A \eor \enot A)}\AS
	\open
	\hypo{3}{A}\AS
	\ellipsesline
	\have[m][-3]{5}{\ered}
	\close
	\have{6}{\enot A}\ni{3-(5)}
	\have{7}{A \eor \enot A}\oi{6}
	\have{8}{\ered}\ne{2,7}
	\close
	\have{9}{A \eor \enot A}\ip{2-8}
\end{fitchproof}
Inside this new subproof, we again need to justify $\ered$. The best way to do this is to work forward from a negated sentence; $\enot(A \eor \enot A)$ on line~$1$ is the only negated sentence we can use. The corresponding unnegated sentence, $A \eor \enot A$, however, directly follows from $A$ (which we have on line~$2$) by $\eor$I. Our complete proof is:
\begin{fitchproof}
	\open
	\hypo{2}{\enot (A \eor \enot A)}\AS
	\open
	\hypo{3}{A}\AS
	\have{4}{A \eor \enot A}\oi{3}
	\have{5}{\ered}\ne{2,4}
	\close
	\have{6}{\enot A}\ni{3-5}
	\have{7}{A \eor \enot A}\oi{6}
	\have{8}{\ered}\ne{2,7}
	\close
	\have{9}{A \eor \enot A}\ip{2-8}
\end{fitchproof}

\practiceproblems

\problempart
Use the strategies to find proofs for each of the following arguments:
\begin{compactlist}
\item $A \eif B, A \eif C \therefore A \eif (B \eand C)$
\item $(A \eand B) \eif C \therefore A \eif (B \eif C)$
\item $A \eif (B \eif C) \therefore (A \eif B) \eif (A \eif C)$
\item $A \eor (B \eand C) \therefore (A \eor B) \eand (A \eor C)$
\item $(A \eand B) \eor (A \eand C) \therefore A \eand (B \eor C)$
\item $A \eor B, A \eif C, B \eif D \therefore C \eor D$
\item $\enot A \lor \enot B \therefore \enot(A \eand B)$
\item $A \eand \enot B \therefore \enot(A \eif B)$
\end{compactlist}

\problempart
Formulate strategies for working backward and forward from $\metav{A} \eiff \metav{B}$.

\problempart
Use the strategies to find proofs for each of the following sentences:
\begin{compactlist}
\item $\enot A \eif (A \eif \ered)$
\item $\enot(A \eand \enot A)$
\item $[(A \eif C) \eand (B \eif C)] \eif [(A \lor B) \eif C]$
\item $\enot(A \eif B) \eif (A \eand \enot B)$
\item $(\enot A \eor B) \eif (A \eif B)$
\end{compactlist}
Since these should be proofs of sentences from no premises, you will start with the respective sentence at the \emph{bottom} of the proof, which will have no premises.

\problempart
Use the strategies to find proofs for each one of the following arguments and sentences:
\begin{compactlist}
\item $\enot\enot A \eif A$
\item $\enot A \eif \enot B \therefore B \eif A$
\item $A \eif B \therefore \enot A \eor B$
\item $\enot(A \eand B) \eif (\enot A \eor \enot B)$
\item $A \eif (B \eor C) \therefore (A \eif B) \eor (A \eif C)$
\item $(A \eif B) \lor (B \eif A)$
\item $((A \eif B) \eif A) \eif A$
\end{compactlist}
These all will require the IP strategy. The last three especially are quite hard!

\chapter{Additional rules for TFL}\label{s:Further}
In \cref{s:BasicTFL}, we introduced the basic rules of our proof system for TFL. In this section, we will add some additional rules to our system. Our extended proof system is a bit easier to work with. (However, in \cref{s:Derived} we will see that they are not strictly speaking \emph{necessary}.)

% \section{Reiteration}
% The first additional rule is \emph{reiteration} (R). This just allows us to repeat ourselves:
% \factoidbox{\begin{fitchproof}
% 	\have[m]{a}{\metav{A}}
% 	\have[\ ]{b}{\metav{A}} \by{R}{a}
% \end{fitchproof}}
% Such a rule is obviously legitimate; we could have used it in our proof in \cref{sec:backred}:
% \begin{fitchproof}
% 	\hypo{1}{A \eor B}
% 	\hypo{7}{\enot A}
% 	\open
% 	\hypo{2}{A} 
% 	\have{8}{\ered}\ne{7,2} 
% 	\have{3}{B}\re{8}
% 	\close 
% 	\open
% 	\hypo{4}{B}
% 	\have{5}{B}\by{R}{4}
% 	\close
% 	\have{6}{B}\oe{1,2-3,4-5} 
% \end{fitchproof}
% This is a fairly typical use of the R rule.

\section{Disjunctive syllogism}
Here is a very natural argument form.
	\begin{earg}
		\item Elizabeth is either in Massachusetts or in DC. 
		\item She is not in DC. 
		\item[\texttherefore] She is in Massachusetts.
	\end{earg}
This is called \emph{disjunctive syllogism}. We add it to our proof system as follows:
\factoidbox{\begin{fitchproof}
	\have[m]{ab}{\metav{A} \eor \metav{B}}
	\have[n]{nb}{\enot \metav{A}}
	\have[\ ]{con}{\metav{B}}\by{DS}{ab, nb}
\end{fitchproof}}
and
\factoidbox{\begin{fitchproof}
	\have[m]{ab}{\metav{A} \eor \metav{B}}
	\have[n]{nb}{\enot \metav{B}}
	\have[\ ]{con}{\metav{A}}\by{DS}{ab, nb}
\end{fitchproof}}
As usual, the disjunction and the negation of one disjunct may occur in either order and need not be adjacent. However, we always cite the disjunction first.

\section{Modus tollens}
Another useful pattern of inference is embodied in the following argument:
	\begin{earg}
		\item If Hilary has won the election, then she is in the White
		House. 
		\item She is not in the White House. 
		\item[\texttherefore] She has not won the election.
	\end{earg}
This inference pattern is called \emph{modus tollens}. The corresponding rule is:
\factoidbox{\begin{fitchproof}
	\have[m]{ab}{\metav{A}\eif\metav{B}}
	\have[n]{a}{\enot\metav{B}}
	\have[\ ]{b}{\enot\metav{A}}\mt{ab,a}
\end{fitchproof}}
As usual, the premises may occur in either order, but we always cite the conditional first.

\section{Double-negation elimination}
Another useful rule is \emph{double-negation elimination}. It does exactly what it says on the tin:
\factoidbox{\begin{fitchproof}
		\have[m]{dna}{\enot \enot \metav{A}}
		\have[ ]{a}{\metav{A}}\dne{dna}
	\end{fitchproof}}
The justification for this is that, in natural language, double-negations tend to cancel out.

That said, you should be aware that context and emphasis can prevent them from doing so. Consider: `Jane is not \emph{not} happy'. Arguably, one cannot infer `Jane is happy', since the first sentence should be understood as meaning the same as  `Jane is not \emph{un}happy'. This is compatible with `Jane is in a state of profound indifference'. As usual, moving to TFL forces us to sacrifice certain nuances of English expressions.

\section{Excluded middle}

Suppose that we can show that if it's sunny outside, then Bill will have brought an umbrella (for fear of burning). Suppose we can also show that, if it's not sunny outside, then Bill will have brought an umbrella (for fear of rain). Well, there is no third way for the weather to be. So, \emph{whatever the weather}, Bill will have brought an umbrella.

This line of thinking motivates the following rule:
\factoidbox{\begin{fitchproof}
		\open
			\hypo[i]{a}{\metav{A}}\AS
			\have[j]{c1}{\metav{B}}
		\close
		\open
			\hypo[k]{b}{\enot\metav{A}}\AS
			\have[l]{c2}{\metav{B}}
		\close
		\have[\ ]{ab}{\metav{B}}\tnd{a-c1,b-c2}
	\end{fitchproof}}
The rule is sometimes called the law of \emph{excluded middle}, since it encapsulates the idea that \metav{A} may be true or $\enot \metav{A}$ may be true, but there is no middle way where neither is true.\footnote{You may sometimes find logicians or philosophers talking about ``tertium non datur.'' That's the same principle as excluded middle; it means ``no third way.'' Logicians who have qualms about indirect proof also have qualms about LEM.} There can be as many lines as you like between $i$ and $j$, and as many lines as you like between $k$ and $l$. Moreover, the subproofs can come in any order, and the second subproof does not need to come immediately after the first.

To see the rule in action, consider:
	$$P \therefore (P \eand D) \eor (P \eand \enot D)$$
Here is a proof corresponding with the argument:
	\begin{fitchproof}
		\hypo{a}{P}\PR
		\open
			\hypo{b}{D}\AS
			\have{ab}{P \eand D}\ai{a, b}
			\have{abo}{(P \eand D) \eor (P \eand \enot D)}\oi{ab}
		\close
		\open
			\hypo{nb}{\enot D}\AS
			\have{anb}{P \eand \enot D}\ai{a, nb}
			\have{anbo}{(P \eand D) \eor (P \eand \enot D)}\oi{anb}
		\close
		\have{con}{(P \eand D) \eor (P \eand \enot D)}\tnd{b-abo, nb-anbo}
	\end{fitchproof}
Here is another example:
\begin{fitchproof}
	\hypo{ana}{A \eif \enot A}\PR
	\open
		\hypo{a}{A}\AS
		\have{na}{\enot A}\ce{ana, a}
	\close
	\open
		\hypo{na1}{\enot A}\AS
		\have{na2}{\enot A}\by{R}{na1}
	\close
	\have{na3}{\enot A}\tnd{a-na, na1-na2}
\end{fitchproof}


\section{De Morgan Rules}
Our final additional rules are called De~Morgan's Laws (named after Augustus De~Morgan). The shape of the rules should be familiar from truth tables.

The first De Morgan rule is:
\factoidbox{\begin{fitchproof}
	\have[m]{ab}{\enot (\metav{A} \eand \metav{B})}
	\have[\ ]{dm}{\enot \metav{A} \eor \enot \metav{B}}\dem{ab}
\end{fitchproof}}
The second De Morgan is the reverse of the first:
\factoidbox{\begin{fitchproof}
	\have[m]{ab}{\enot \metav{A} \eor \enot \metav{B}}
	\have[\ ]{dm}{\enot (\metav{A} \eand \metav{B})}\dem{ab}
\end{fitchproof}}
The third De Morgan rule is the \emph{dual} of the first:
\factoidbox{\begin{fitchproof}
	\have[m]{ab}{\enot (\metav{A} \eor \metav{B})}
	\have[\ ]{dm}{\enot \metav{A} \eand \enot \metav{B}}\dem{ab}
\end{fitchproof}}
And the fourth is the reverse of the third:
\factoidbox{\begin{fitchproof}
	\have[m]{ab}{\enot \metav{A} \eand \enot \metav{B}}
	\have[\ ]{dm}{\enot (\metav{A} \eor \metav{B})}\dem{ab}
\end{fitchproof}}
\emph{These are all of the additional rules of our proof system for TFL.}

\practiceproblems
\solutions
\problempart
\label{pr.justifyTFLproof}
The following proofs are missing their citations (rule and line numbers). Add them wherever they are required:
\begin{compactlist}
\item\begin{fitchproof}
\hypo{1}{W \eif \enot B}
\hypo{2}{A \eand W}
\hypo{2b}{B \eor (J \eand K)}
\have{3}{W}{}
\have{4}{\enot B} {}
\have{5}{J \eand K} {}
\have{6}{K}{}
\end{fitchproof}
\item\begin{fitchproof}
\hypo{1}{L \eiff \enot O}
\hypo{2}{L \eor \enot O}
\open
	\hypo{a1}{\enot L}
	\have{a2}{\enot O}{}
	\have{a3}{L}{}
	\have{a4}{\ered}{}
\close
\have{3b}{\enot\enot L}{}
\have{3}{L}{}
\end{fitchproof}
\item\begin{fitchproof}
\hypo{1}{Z \eif (C \eand \enot N)}
\hypo{2}{\enot Z \eif (N \eand \enot C)}
\open
	\hypo{a1}{\enot(N \eor  C)}
	\have{a2}{\enot N \eand \enot C} {}
	\have{a6}{\enot N}{}
	\have{b4}{\enot C}{}
		\open
		\hypo{b1}{Z}
		\have{b2}{C \eand \enot N}{}
		\have{b3}{C}{}
		\have{red}{\ered}{}
	\close
	\have{a3}{\enot Z}{}
	\have{a4}{N \eand \enot C}{}
	\have{a5}{N}{}
	\have{a7}{\ered}{}
\close
\have{3b}{\enot\enot(N \eor C)}{}
\have{3}{N \eor C}{}
\end{fitchproof}
\end{compactlist}

\problempart 
Give a proof for each of these arguments:
\begin{compactlist}
\item $E\eor F$, $F\eor G$, $\enot F \therefore E \eand G$
\item $M\eor(N\eif M) \therefore \enot M \eif \enot N$
\item $(M \eor N) \eand (O \eor P)$, $N \eif P$, $\enot P \therefore M\eand O$
\item $(X\eand Y)\eor(X\eand Z)$, $\enot(X\eand D)$, $D\eor M \therefore M$
\end{compactlist}



\chapter{Proof-theoretic concepts}\label{s:ProofTheoreticConcepts}

In this chapter we will introduce some new vocabulary. The following expression:
$$\metav{A}_1, \metav{A}_2, \ldots, \metav{A}_n \proves \metav{C}$$
means that there is some proof which ends with $\metav{C}$ whose undischarged assumptions are among $\metav{A}_1, \metav{A}_2, \ldots, \metav{A}_n$. When we want to say that it is \emph{not} the case that there is some proof which ends with $\metav{C}$ from $\metav{A}_1$, $\metav{A}_2$, \dots,~$\metav{A}_n$, we write:  $$\metav{A}_1, \metav{A}_2, \ldots, \metav{A}_n \nproves \metav{C}$$ 

The symbol `$\proves$' is called the \emph{single turnstile}. We want to emphasize that this is not the {double turnstile} symbol (`$\entails$') that we introduced in \cref{s:SemanticConcepts} to symbolize entailment. The single turnstile, `$\proves$', concerns the existence of proofs; the double turnstile, `$\entails$', concerns the existence of valuations (or interpretations, when used for FOL). \emph{They are very different notions.}

Armed with our `$\proves$' symbol, we can introduce some more terminology. To say that there is a proof of $\metav{A}$ with no undischarged assumptions, we write: ${} \proves \metav{A}$. In this case, we say that $\metav{A}$ is a \define{theorem}.
	\factoidbox{\label{def:syntactic_tautology_in_sl}
		$\metav{A}$ is a \define{theorem} \ifeff{} $\proves \metav{A}$
	}
\newglossaryentry{theorem}
{
name=theorem,
description={A sentence that can be proved without any premises}
}

To illustrate this, suppose we want to show that `$\enot (A \eand \enot A)$' is a theorem.  So we need a proof of `$\enot(A \eand \enot A)$' which has \emph{no} undischarged assumptions. However, since we want to prove a sentence whose main logical operator is a negation, we will want to start with a \emph{subproof} within which we assume `$A \eand \enot A$', and show that this assumption leads to contradiction. All told, then, the proof looks like this:
	\begin{fitchproof}
		\open
			\hypo{con}{A \eand \enot A}\AS
			\have{a}{A}\ae{con}
			\have{na}{\enot A}\ae{con}
			\have{red}{\ered}\ne{na, a}
		\close
		\have{lnc}{\enot (A \eand \enot A)}\ni{con-red}
	\end{fitchproof}
We have therefore proved `$\enot (A \eand \enot A)$' on no (undischarged) assumptions. This particular theorem is an instance of what is sometimes called \emph{the Law of Non-Contradiction}.

To show that something is a theorem, you just have to find a suitable proof. It is typically much harder to show that something is \emph{not} a theorem. To do this, you would have to demonstrate, not just that certain proof strategies fail, but that \emph{no} proof is possible. Even if you fail in trying to prove a sentence in a thousand different ways, perhaps the proof is just too long and complex for you to make out. Perhaps you just didn't try hard enough.

Here is another new bit of terminology:
	\factoidbox{
		Two sentences \metav{A} and \metav{B} are \define{provably equivalent} \ifeff{} each can be proved from the other; i.e., both $\metav{A}\proves\metav{B}$ and $\metav{B}\proves\metav{A}$.
	}
        
\newglossaryentry{provably equivalent}
{
  name=provable equivalence,
  text = provably equivalent,
description={A property held by pairs of statements if and only if there is a derivation which takes you from each one to the other one}
}


As in the case of showing that a sentence is a theorem, it is relatively easy to show that two sentences are provably equivalent: it just requires a pair of proofs. Showing that sentences are \emph{not} provably equivalent would be much harder: it is just as hard as showing that a sentence is not a theorem.

Here is a third, related, bit of terminology:
	\factoidbox{
	The sentences $\metav{A}_1,\metav{A}_2,\ldots, \metav{A}_n$ are
	\define{jointly inconsistent} \ifeff{} the contradiction~$\ered$ can be proved from
	them, i.e., $\metav{A}_1,\metav{A}_2,\ldots, \metav{A}_n \proves
	\ered$. If they are not \define{inconsistent}, we call them
	\define{jointly consistent}.
	}
        
\newglossaryentry{inconsistent}
{    name={inconsistency}, 
  description={Sentences are inconsistent \ifeff{} the contradiction~$\ered$ can be proved from them},
    text={inconsistent}
}

\newglossaryentry{consistent}
{    name={consistency}, 
  description={Sentences are jointly consistent \ifeff{} the contradiction~$\ered$ can\emph{not} be proved from them},
    text={consistent}
}

It is easy to show that some sentences are inconsistent: you just need
to prove the contradiction~$\ered$, assuming all the sentences as
premises. Showing that some sentences are \emph{not} inconsistent is much
harder. It would require more than just providing a proof or two; it
would require showing that no proof of a certain kind is
\emph{possible}.

\
\\
This table summarizes whether one or two proofs suffice, or whether we must reason about all possible proofs.

\begin{center}
\begin{tabular}{l l l}
%\cline{2-3}
 & \textbf{Yes} & \textbf{No}\\
 \hline
%\cline{2-3}
theorem? & one proof & all possible proofs\\
inconsistent? &  one proof  & all possible proofs\\
equivalent? & two proofs & all possible proofs\\
consistent? & all possible proofs & one proof\\
\end{tabular}
\end{center}


\practiceproblems
\problempart
Show that each of the following sentences is a theorem:
\begin{compactlist}
\item $O \eif O$
\item $N \eor \enot N$
\item $J \eiff [J\eor (L\eand\enot L)]$
\item $((A \eif B) \eif A) \eif A$ 
\end{compactlist}

\problempart
Provide proofs to show each of the following:
\begin{compactlist}
\item $C\eif(E\eand G), \enot C \eif G \proves G$
\item $M \eand (\enot N \eif \enot M) \proves (N \eand M) \eor \enot M$
\item $(Z\eand K)\eiff(Y\eand M), D\eand(D\eif M) \proves Y\eif Z$
\item $(W \eor X) \eor (Y \eor Z), X\eif Y, \enot Z \proves W\eor Y$
\end{compactlist}

\problempart
Show that each of the following pairs of sentences are provably equivalent:
\begin{compactlist}
\item $R \eiff E$, $E \eiff R$
\item $G$, $\enot\enot\enot\enot G$
\item $T\eif S$, $\enot S \eif \enot T$
\item $U \eif I$, $\enot(U \eand \enot I)$
\item $\enot (C \eif D), C \eand \enot D$
\item $\enot G \eiff H$, $\enot(G \eiff H)$ 
\end{compactlist}

\problempart
If you know that $\metav{A}\proves\metav{B}$, what can you say about $(\metav{A}\eand\metav{C})\proves\metav{B}$? What about $(\metav{A}\eor\metav{C})\proves\metav{B}$? Explain your answers.

\

\problempart In this chapter, we claimed that it is just as hard to show that two sentences are not provably equivalent, as it is to show that a sentence is not a theorem. Why did we claim this? (\emph{Hint}: think of a sentence that would be a theorem \ifeff{} \metav{A} and \metav{B} were provably equivalent.)


\chapter{Derived rules}\label{s:Derived}
In this section, we will see why we introduced the rules of our proof system in two separate batches. In particular, we want to show that the additional rules of \cref{s:Further} are not strictly speaking necessary, but can be derived from the basic rules of \cref{s:BasicTFL}.

\section{Derivation of Reiteration}
To illustrate what it means to derive a \emph{rule} from other rules, first consider reiteration. It is a basic rule of our system, but it is also not necessary. Suppose you have some sentence on some line of your deduction:
\begin{fitchproof}
	\have[m]{a}{\metav{A}}
\end{fitchproof}
You now want to repeat yourself, on some line $k$. You could just invoke the rule~R. But equally well, you can do this with other basic rules of \cref{s:BasicTFL}:
\begin{fitchproof}
	\have[m]{a}{\metav{A}}
	\have[k]{aa}{\metav{A} \eand \metav{A}}\ai{a, a}
	\have{a2}{\metav{A}}\ae{aa}
\end{fitchproof}
To be clear: this is not a proof. Rather, it is a proof \emph{scheme}. After all, it uses a variable, `$\metav{A}$', rather than a sentence of TFL, but the point is simple: Whatever sentences of TFL we plugged in for `$\metav{A}$', and whatever lines we were working on, we could produce a bona fide proof. So you can think of this as a recipe for producing proofs.

Indeed, it is a recipe which shows us the following: anything we can prove using the rule R, we can prove (with one more line) using just the basic rules of \cref{s:BasicTFL} without~R. That is what it means to say that the rule~R can be derived from the other basic rules: anything that can be justified using~R can be justified using only the other basic rules.


\section{Derivation of Disjunctive Syllogism}
Suppose that you are in a proof, and you have something of this form:
\begin{fitchproof}
	\have[m]{ab}{\metav{A}\eor\metav{B}}
	\have[n]{na}{\enot \metav{A}}
\end{fitchproof}
You now want, on line $k$, to prove $\metav{B}$. You can do this with the rule of DS, introduced in \cref{s:Further}, but equally well, you can do this with the \emph{basic} rules of \cref{s:BasicTFL}:
	\begin{fitchproof}
		\have[m]{ab}{\metav{A}\eor\metav{B}}
		\have[n]{na}{\enot \metav{A}}
		\open
			\hypo[k]{a}{\metav{A}}\AS
			\have{red}{\ered}\ne{na, a}
			\have{b1}{\metav{B}}\re{red}
		\close
		\open
			\hypo{b}{\metav{B}}\AS
			\have{b2}{\metav{B}}\by{R}{b}
		\close
	\have{con}{\metav{B}}\oe{ab, a-b1, b-b2}
\end{fitchproof}
So the DS rule, again, can be derived from our more basic rules. Adding it to our system did not make any new proofs possible. Anytime you use the DS rule, you could always take a few extra lines and prove the same thing using only our basic rules. It is a \emph{derived} rule.

\section{Derivation of Modus Tollens}
Suppose you have the following in your proof:
\begin{fitchproof}
	\have[m]{ab}{\metav{A}\eif\metav{B}}
	\have[n]{a}{\enot\metav{B}}
\end{fitchproof}
You now want, on line $k$, to prove $\enot \metav{A}$. You can do this with the rule of MT, introduced in \cref{s:Further}. Equally well, you can do this with the \emph{basic} rules of \cref{s:BasicTFL}:
\begin{fitchproof}
	\have[m]{ab}{\metav{A}\eif\metav{B}}
	\have[n]{nb}{\enot\metav{B}}
		\open
		\hypo[k]{a}{\metav{A}}\AS
		\have{b}{\metav{B}}\ce{ab, a}
		\have{nb1}{\ered}\ne{nb, b}
		\close
	\have{no}{\enot\metav{A}}\ni{a-nb1}
\end{fitchproof}
Again, the rule of MT can be derived from the \emph{basic} rules of \cref{s:BasicTFL}.

\section{Derivation of Double-Negation Elimination}
Consider the following deduction scheme:
	\begin{fitchproof}
	\have[m]{m}{\enot \enot \metav{A}}
	\open
		\hypo[k]{a}{\enot\metav{A}}\AS
		\have{a1}{\ered}\ne{m, a}
	\close
	\have{con}{\metav{A}}\ip{a-a1}
\end{fitchproof}
Again,  we can derive the DNE rule from the \emph{basic} rules of \cref{s:BasicTFL}.

\section{Derivation of Excluded Middle}
Suppose you want to prove something using the LEM rule, i.e., you have in your proof
\begin{fitchproof}
  \open
  \hypo[m]{a}{\metav{A}}\AS
  \have[n]{aaa}{\metav{B}}
  \close
  \open
  \hypo[k]{b}{\enot\metav{A}}\AS
  \have[l]{bbb}{\metav{B}}
  \close
\end{fitchproof}
You now want, on line $l+1$, to prove $\metav{B}$. The rule LEM from
\cref{s:Further} would allow you to do it. But can you do this with
the \emph{basic} rules of \cref{s:BasicTFL}?

One option is to first prove $\metav{A} \eor \enot\metav{A}$, and then apply $\eor$E, i.e., proof by cases:
\begin{fitchproof}
  \open
  \hypo[m]{a}{\metav{A}}\AS
  \have[n]{aaa}{\metav{B}}
  \close
  \open
  \hypo[k]{b}{\enot\metav{A}}\AS
  \have[l]{bbb}{\metav{B}}
  \close
  \ellipsesline
  \have[i]{tnd}{\metav{A} \eor \enot \metav{A}}
  \have[i+1]{fin}{\metav{B}}\oe{tnd, a-aaa,b-bbb}
\end{fitchproof}
(We gave a proof of $\metav{A} \eor \enot\metav{A}$ using only our basic rules in \cref{s:proofLEM}.)

Here is another way that is a bit more complicated than the ones before. What you have to do is embed your two subproofs inside another subproof. The assumption of the subproof will be $\enot \metav{B}$, and the last line will be $\ered$. Thus, the complete subproof is the kind you need to conclude \metav{B} using IP. Inside the proof, you'd have to do a bit more work to get~$\ered$:
\begin{fitchproof}
  \open
  \hypo[m]{nb}{\enot\metav{B}}\AS
  \open
  \hypo{a}{\metav{A}}\AS
  \ellipsesline
  \have[n]{aaa}{\metav{B}}
  \have{aaaa}{\ered}\ne{nb, aaa}
  \close
  \open
  \hypo{b}{\enot\metav{A}}\AS
  \ellipsesline
  \have[l]{bbb}{\metav{B}}
  \have{bbbb}{\ered}\ne{nb, bbb}
  \close
  \have{na}{\enot\metav{A}}\ni{(a)-(aaaa)}
  \have{nna}{\enot\enot\metav{A}}\ni{(b)-(bbbb)}
  \have{bot}{\ered}\ne{nna, na}
  \close
  \have{B}{\metav{B}}\ip{nb-(bot)}
\end{fitchproof}
Note that because we add an assumption at the top and additional conclusions inside the subproofs, the line numbers change. You may have to stare at this for a while before you understand what's going on.


\section{Derivation of De Morgan rules}
Here is a demonstration of how we could derive the first De Morgan rule:
 	\begin{fitchproof}
		\have[m]{nab}{\enot (\metav{A} \eand \metav{B})}
		\open
			\hypo[k]{a}{\metav{A}}\AS
			\open
				\hypo{b}{\metav{B}}\AS
				\have{ab}{\metav{A} \eand \metav{B}}\ai{a,b}
				\have{nab1}{\ered}\ne{nab, ab}
			\close
			\have{nb}{\enot \metav{B}}\ni{(b)-(nab1)}
			\have{dis}{\enot\metav{A} \eor \enot \metav{B}}\oi{nb}
		\close
		\open
			\hypo{na1}{\enot \metav{A}}\AS
			\have{dis1}{\enot\metav{A} \eor \enot \metav{B}}\oi{na1}
		\close
		\have{con}{\enot \metav{A} \eor \enot \metav{B}}\tnd{a-(dis), (na1)-(dis1)}
	\end{fitchproof}
Here is a demonstration of how we could derive the second De Morgan rule:
 	\begin{fitchproof}
		\have[m]{nab}{\enot \metav{A} \eor \enot \metav{B}}
		\open
			\hypo[k]{ab}{\metav{A} \eand \metav{B}}\AS
			\have{a}{\metav{A}}\ae{ab}
			\have{b}{\metav{B}}\ae{ab}
			\open
				\hypo{na}{\enot \metav{A}}\AS
				\have{c1}{\ered}\ne{na, a}
			\close
			\open
				\hypo{nb}{\enot \metav{B}}\AS
				\have{c2}{\ered}\ne{nb, b}
			\close
			\have{con2}{\ered}\oe{nab, (na)-(c1), (nb)-(c2)}
		\close
		\have{nab}{\enot (\metav{A} \eand \metav{B})}\ni{ab-(con2)}
	\end{fitchproof}
Similar demonstrations can be offered explaining how we could derive the third and fourth De Morgan rules. These are left as exercises.



\practiceproblems

\problempart
Provide proof schemes that justify the addition of the third and fourth De Morgan rules as derived rules.

\

\problempart
The proofs you offered in response to the practice exercises of \crefrange{s:Further}{s:ProofTheoreticConcepts} used derived rules. Replace the use of derived rules, in such proofs, with only basic rules. You will find some `repetition' in the resulting proofs; in such cases, offer a streamlined proof using only basic rules.  (This will give you a sense, both of the power of derived rules, and of how all the rules interact.)

\

\problempart
Give a proof of $\metav{A} \eor \enot\metav{A}$. Then
give a proof that \emph{uses only the basic rules}.

\

\problempart
Show that if you had LEM as a basic rule, you could justify IP as a derived rule. That is, suppose you had the proof:
\begin{fitchproof}
  \open
  \hypo[m]{a}{\enot\metav{A}}\AS
  \have[\ ]{aa}{\dots}
  \have[n]{aaa}{\ered}
  \close
\end{fitchproof}
How could you use it to prove \metav{A} without using IP but with using LEM as well as all the other basic rules?

\

\problempart
Give a proof of the first De Morgan rule, but using only the basic rules, in particular, \emph{without using LEM}. (Of course, you can combine the proof using LEM with the proof \emph{of}~LEM. Try to find a proof directly.)

\chapter{Soundness and completeness}
\label{sec:soundness_and_completeness}

In \cref{s:ProofTheoreticConcepts}, we saw that we could use derivations to test for the same concepts we used truth tables to test for. Not only could we use derivations to prove that an argument is valid, we could also use them to test if a sentence is a tautology or a pair of sentences are equivalent. We also started using the single turnstile the same way we used the double turnstile. If we could prove that \metav{A} was a tautology using a truth table, we wrote $\entails \metav{A}$, and if we could prove it using a derivation, we wrote $\proves \metav{A}$.

You may have wondered at that point if the two kinds of turnstiles always worked the same way. If you can show that \metav{A} is a tautology using truth tables, can you also always show that it is a theorem using a derivation? Is the reverse true? Are these things also true for valid arguments and pairs of equivalent sentences? As it turns out, the answer to all these questions and many more like them is yes. We can show this by defining all these concepts separately and then proving them equivalent. That is, we imagine that we actually have two notions of validity, valid$_{\entails}$ and  valid$_{\proves}$ and then show that the two concepts always work the same way.

To begin with, we need to define all of our logical concepts separately for truth tables and derivations. A lot of this work has already been done. We handled all of the truth table definitions in \cref{s:SemanticConcepts}. We have also already given proof-theoretic definitions for theorems and pairs of logically equivalent sentences. The other definitions follow naturally. For most logical properties we can devise a test using derivations, and those that we cannot test for directly can be defined in terms of the concepts that we can define.

For instance, we defined a theorem as a sentence that can be derived
without any premises (p.~\pageref{def:syntactic_tautology_in_sl}).
Since the negation of a contradiction is a tautology, we can define an
\define{inconsistent sentence in TFL}
\label{def:syntactic_contradiction_in_sl} as a sentence whose negation
can be derived without any premises.\footnote{Note that $\lnot
\metav{A}$ is a theorem \ifeff{} $\metav{A} \proves \ered$.} The syntactic
definition of a contingent sentence is a little different. We don't
have any practical, finite method for proving that a sentence is
contingent using derivations, the way we did using truth tables. So we
have to content ourselves with defining ``contingent sentence''
negatively. A sentence is \define{{proof-theoretically contingent in
TFL}} \label{def:syntactically_contingent_in_sl} if it is neither a
theorem nor an inconsistent sentence.
 

A collection of sentences is \define{inconsistent in TFL} \label{def:syntactically_inconsistent_ in_sl} if and only if one can derive the contradiction~$\ered$ from them. Consistency, on the other hand, is like contingency, in that we do not have a practical finite method to test for it directly. So again, we have to define a term negatively. A collection of sentences is \define{consistent in TFL} \label{def:syntactically consistent in SL} if and only if they are not inconsistent.
    

Finally, an argument is \define{provably valid in TFL} \label{def:syntactically_valid_in_SL} if and only if there is a derivation of its conclusion from its premises. All of these definitions are given in \cref{table:truth_tables_or_derivations}.

\def\tmptable{\textbf{Concept} 		&	\textbf{Truth table (semantic) definition} 	&	\textbf{Proof-theoretic (syntactic) definition} \\ \hline \hline

Tautology/ theorem   &	A sentence whose truth table only has Ts under the main connective & A sentence that can be derived without any premises.	 \\ \hline
 
Contradiction/ inconsistent sentence		&	A sentence whose truth table only has Fs under the main connective  &	A sentence whose negation can be derived without any premises\\ \hline

Contingent sentence	&	A sentence whose truth table contains both Ts and Fs under the main connective & A sentence that is not a theorem or contradiction \\ \hline

Equivalent sentences &	The columns under the main connectives are identical.& The sentences can be derived from each other	\\ \hline

Unsatisfiable/ inconsistent sentences	&	Sentences which do not have a single line in their truth table where they are all true.	& Sentences  from which one can derive the contradiction~$\ered$ \\ \hline

Satisfiable/ Consistent sentences	&	Sentences which have at least one line in their truth table where they are all true. & Sentences from which one cannot derive the contradiction~$\ered$	\\ \hline

Valid argument		&	An argument whose truth table has no lines where there are all Ts under main connectives for the premises and an F under the main connective for the conclusion.  & An argument where one can derive the conclusion from the premises.}

\ifHTMLtarget
\begin{table}
\begin{tabularx}{\textwidth}{X||X|X}
\tmptable
\end{tabularx}
\caption{Two ways to define logical concepts.}
\label{table:truth_tables_or_derivations}
\end{table}
\else
\begin{sidewaystable}\small
\tabulinesep=1ex
\begin{tabu}{X[.5,c,m] ||X[1,l,m] |X[1,l,m]}
\tmptable
\end{tabu}
\caption{Two ways to define logical concepts.}
\label{table:truth_tables_or_derivations}
\end{sidewaystable}
\fi

All of our concepts have now been defined both semantically (using
valuations and truth tables) and proof-theoretically (on the basis of
natural deduction). How can we establish that these definitions always
work the same way? A full proof here goes well beyond the scope of
this book. However, we can sketch what it would be like. We will focus
on showing the two notions of validity to be equivalent.  From that
the other concepts will follow quickly. The proof will have to go in
two directions. First we will have to show that things which are
proof-theoretically valid will also be semantically valid. In other words,
everything that we can prove using derivations could also be proven
using truth tables. Put symbolically, we want to show that
valid$_{\proves}$ implies valid$_{\entails}$. Afterwards, we will need
to show things in the other directions,  valid$_{\entails}$ implies
valid$_{\proves}$

\newglossaryentry{soundness}
{
name=soundness,
description={A property held by logical systems if and only if $\proves $ implies $\entails $}
}

This argument from $\proves $ to $\entails $ is the problem of
\define{\gls{soundness}}. \label{def:soundness} A proof system is
\define{sound} if there are no derivations of arguments that can be
shown invalid by truth tables. \label{def_Soundness} Demonstrating
that the proof system is sound would require showing that \emph{any}
possible proof is the proof of a valid argument. It would not be
enough simply to succeed when trying to prove many valid arguments and
to fail when trying to prove invalid ones. The proof itself requires
some care. In a nutshell, it involves showing that every inference
rule preserves the property that its conclusion is entailed by the
premises of the proof together with whatever assumptions are open. You
can find the details of this proof worked out in \cref{ch:Soundness}. 

Soundness means that $\metav{A} \proves \metav{B}$ implies $\metav{A}
\entails \metav{B}.$ What about the other direction, that is, why
think that \emph{every} argument that can be shown valid using truth
tables can also be proven using a derivation? 

\newglossaryentry{completeness}
{
name=completeness,
description={A property held by logical systems if and only if $\entails $ implies $\proves $}
}

This is the problem of completeness. A proof system has the property of  \define{\gls{completeness}} \label{def:completeness} if and only if there is a derivation of every semantically valid argument. Proving that a system is complete is generally harder than proving that it is sound. Proving that a system is sound amounts to showing that all of the rules of your proof system work the way they are supposed to. Showing that a system is complete means showing that you have included \emph{all} the rules you need, that you haven't left any out. Showing this is beyond the scope of this book. The important point is that, happily, the proof system for TFL is both sound and complete. This is not the case for all proof systems or all formal languages. Because it is true of TFL, we can choose to give proofs or give truth tables---whichever is easier for the task at hand.

Now that we know that the truth table method is interchangeable with the method of derivations, you can chose which method you want to use for any given problem. Students often prefer to use truth tables, because they can be produced purely mechanically, and that seems `easier'. However, we have already seen that truth tables become impossibly large after just a few sentence letters. On the other hand, there are a couple situations where using proofs simply isn't possible. We syntactically defined a contingent sentence as a sentence that couldn't be proven to be a tautology or a contradiction. There is no practical way to prove this kind of negative statement. We will never know if there isn't some proof out there that a statement is a contradiction and we just haven't found it yet. We have nothing to do in this situation but resort to truth tables. Similarly, we can use derivations to prove two sentences equivalent, but what if we want to prove that they are \emph{not} equivalent? We have no way of proving that we will never find the relevant proof. So we have to fall back on truth tables again.

\Cref{table.ProofOrModel} summarizes when it is best to give proofs 
and when it is best to give truth tables.

\def\tmptable{\textbf{Logical property} 	&	\textbf{To prove it present} 	&	\textbf{To prove it absent} \\ \hline \hline
Being a theorem 		& Derive the sentence  						& Find a false line in the truth table for the sentence \\ \hline
Being a contradiction 	&  Derive the negation of the sentence  		 & Find a true line in the truth table for the sentence\\ \hline
Contingency 			& Find a false line and a true line in the truth table for the sentence & Prove the sentence or its negation\\ \hline
Equivalence 			& Derive each sentence from the other 		 & Find a line in the truth tables for the sentence where they have different values\\ \hline
Consistency 		& Find a line in truth table for the sentence where they all are true & Derive a contradiction from the sentences\\ \hline
Validity 				& Derive the conclusion from the premises & Find a line in the truth table where the premises are true and the conclusion false. \\ }

\ifHTMLtarget
\begin{table}
\begin{tabularx}{\textwidth}{X||X|X}
\tmptable
\end{tabularx}
\caption{When to provide a truth table and when to provide a proof.}
\label{table.ProofOrModel}
\end{table}
\else
\begin{table}\small
\tabulinesep=1ex
\begin{tabu}{X[.7,c,m] ||X[1,l,m] |X[1,l,m]}
\tmptable
\end{tabu}
\caption{When to provide a truth table and when to provide a proof.}
\label{table.ProofOrModel}
\end{table}
\fi



\practiceproblems
\noindent\problempart Use either a derivation or a truth table for each of the following.
\begin{compactlist}%[label=(\arabic*)]
\item Show that $A \eif [((B \eand C) \eor D) \eif A]$ is a theorem.
\item Show that $A \eif (A \eif B)$ is not a theorem.
\item Show that the sentence $A \eif \enot{A}$ is not a contradiction.
\item Show that the sentence $A \eiff \enot A$ is a contradiction.
\item Show that the sentence $ \enot (W \eif (J \eor J)) $ is contingent.
\item Show that the sentence $ \enot(X \eor (Y \eor Z)) \eor (X \eor (Y \eor Z))$ is not contingent.
 \item Show that the sentence $B \eif \enot S$ is equivalent to the sentence $\enot \enot B \eif \enot S$.
\item Show that the sentence $ \enot (X \eor O) $ is not equivalent to the sentence $X \eand O$.
\item Show that the sentences $\enot(A \eor B)$, $C$, $C \eif A$  are jointly inconsistent.
\item Show that the sentences $\enot(A \eor B)$, $\enot{B}$, $B \eif A$ are jointly consistent.
\item Show that $\enot(A \eor (B \eor C)) $ \therefore $ \enot{C}$ is valid.
\item Show that $\enot(A \eand (B \eor C))$ \therefore $ \enot{C}$ is invalid.
\end{compactlist}


\noindent\problempart Use either a derivation or a truth table for each of the following.
\begin{compactlist}%[label=(\arabic*)]
\item Show that $A \eif (B \eif A)$ is a theorem.
\item Show that $\enot (((N \eiff Q) \eor Q) \eor N)$ is not a theorem.
\item Show that $ Z \eor (\enot Z \eiff Z) $ is contingent.
\item Show that $ (L \eiff ((N \eif N) \eif L)) \eor H $ is not contingent.
\item Show that $ (A \eiff A) \eand (B \eand \enot B)$ is a contradiction.
\item Show that $ (B \eiff (C \eor B)) $ is not a contradiction.
\item Show that $ ((\enot X \eiff X) \eor X) $ is equivalent to $X$.
\item Show that $F \eand (K \eand R) $ is not equivalent to $ (F \eiff (K \eiff R)) $.
\item Show that the sentences $ \enot (W \eif W)$, $(W \eiff W) \eand W$, $E \eor (W \eif \enot (E \eand W))$ are jointly inconsistent.
\item Show that the sentences  $\enot R \eor C $, $(C \eand R) \eif \enot R$, $(\enot (R \eor R) \eif R) $ are jointly consistent.
\item Show that $\enot \enot (C \eiff \enot C), ((G \eor C) \eor G) \therefore ((G \eif C) \eand G) $ is valid.
\item Show that $ \enot \enot L,  (C \eif \enot L) \eif C) \therefore \enot C$ is invalid.
\end{compactlist}

