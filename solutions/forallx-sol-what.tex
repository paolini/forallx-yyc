%!TEX root = forallxyyc-solutions.tex
%\part{Key notions}
%\label{ch.intro}
%\addtocontents{toc}{\protect\mbox{}\protect\hrulefill\par}

\chapter{Arguments}
Highlight the phrase which expresses the conclusion of each of these arguments:
\begin{compactlist}
	\item It is sunny. So \myanswer{I should take my sunglasses}.
	\item \myanswer{It must have been sunny}. I did wear my sunglasses, after all.
	\item No one but you has had their hands in the cookie-jar. And the scene of the crime is littered with cookie-crumbs. \myanswer{You're the culprit}!
	\item Miss Scarlett and Professor Plum were in the study at the time
	of the murder. And Reverend Green had the candlestick in the
	ballroom, and we know that there is no blood on his hands. Hence
	\myanswer{Colonel Mustard did it in the kitchen with the lead
	pipe}. Recall, after all, that the gun had not been fired.
\end{compactlist}

\chapter{The scope of logic}
\problempart
Which of the following arguments is valid? Which is invalid?
\begin{compactlist}
\item\begin{earg}
\item Socrates is a man.
\item All men are carrots.
\item[\texttherefore]  Socrates is a carrot. \hfill \myanswer{Valid}
\end{earg}

\item\begin{earg}
\item Abe Lincoln was either born in Illinois or he was once president.
\item Abe Lincoln was never president.
\item[\texttherefore] Abe Lincoln was born in Illinois. \hfill \myanswer{Valid}
\end{earg}

\item\begin{earg}
\item If I pull the trigger, Abe Lincoln will die.
\item I do not pull the trigger.
\item[\texttherefore] Abe Lincoln will not die. \hfill \myanswer{Invalid \\ Abe Lincoln might die for some other reason: someone else might pull the trigger; he might die of old age.}
\end{earg}

\item\begin{earg}
\item Abe Lincoln was either from France or from Luxemborg.
\item Abe Lincoln was not from Luxemborg.
\item[\texttherefore] Abe Lincoln was from France. \hfill \myanswer{Valid}
\end{earg}

\item\begin{earg}
\item If the world were to end today, then I would not need to get up tomorrow morning.
\item I will need to get up tomorrow morning.
\item[\texttherefore] The world will not end today. \hfill \myanswer{Valid}
\end{earg}

\item\begin{earg}
\item Joe is now 19 years old.
\item Joe is now 87 years old.
\item[\texttherefore] Bob is now 20 years old. \hfill \myanswer{Valid}
\\\myanswer{An argument is valid if and only if it is impossible for all the premises to be true and the conclusion false. It is impossible for all the premises to be true; so it is certainly impossible that the premises are all true and the conclusion is false.}
\end{earg}
\end{compactlist}

\problempart
\label{pr.EnglishCombinations}
Could there be:
	\begin{compactlist}
		\item A valid argument that has one false premise and one true premise? \hfill \myanswer{Yes. \\Example: the first argument, above.}
		\item A valid argument that has only false premises? \hfill \myanswer{Yes.\\Example: Socrates is a frog, all frogs are excellent pianists, therefore Socrates is an excellent pianist.}
		\item A valid argument with only false premises and a false conclusion? \hfill \myanswer{Yes. \\The same example will suffice.}
		\item An invalid argument that can be made valid by the addition of a new premise? \hfill\myanswer{Yes.\\ Plenty of examples, but let me offer a more general observation. We can \emph{always} make an invalid argument valid, by adding a contradiction into the premises. For an argument is valid if and only if it is impossible for all the premises to be true and the conclusion false. If the premises are contradictory, then it is impossible for them all to be true (and the conclusion false).}
		\item A valid argument that can be made invalid by the addition of a new premise? \hfill \myanswer{No.\\ An argument is valid if and only if it is impossible for all the premises to be true and the conclusion false. Adding another premise will only make it harder for the premises all to be true together.}
	\end{compactlist}
In each case: if so, give an example; if not, explain why not.

\chapter{Other logical notions}
\setcounter{ProbPart}{0}
\problempart
\label{pr.EnglishTautology}
For each of the following: Is it necessarily true, necessarily false, or contingent?
\begin{compactlist}
\item Caesar crossed the Rubicon.
\hfill \myanswer{Contingent}
\item Someone once crossed the Rubicon.
\hfill \myanswer{Contingent}
\item No one has ever crossed the Rubicon.
\hfill \myanswer{Contingent}
\item If Caesar crossed the Rubicon, then someone has.
\hfill \myanswer{Necessarily true}
\item Even though Caesar crossed the Rubicon, no one has ever crossed the Rubicon.
\hfill \myanswer{Necessarily false}
\item If anyone has ever crossed the Rubicon, it was Caesar.
\hfill \myanswer{Contingent}
\end{compactlist}

\problempart
For each of the following: Is it a necessary truth, a necessary falsehood, or contingent?
\begin{compactlist}
\item Elephants dissolve in water.
\item Wood is a light, durable substance useful for building things.
\item If wood were a good building material, it would be useful for building things.
\item I live in a three story building that is two stories tall.
\item If gerbils were mammals they would nurse their young.
\end{compactlist}

\problempart Which of the following pairs of sentences are necessarily  equivalent? 

\begin{compactlist}
\item Elephants dissolve in water.	\\
	If you put an elephant in water, it will disintegrate.
\item All mammals dissolve in water.\\		
	If you put an elephant in water, it will disintegrate.
\item George Bush was the 43rd president. \\
	 Barack Obama is the 44th president.
\item Barack Obama is the 44th president. \\
	  Barack Obama was president immediately after the 43rd president.
\item Elephants dissolve in water. 	\\	
	All mammals dissolve in water.
\end{compactlist}

\problempart Which of the following pairs of sentences are necessarily equivalent? 

\begin{compactlist}
\item  Thelonious Monk played piano.	\\
	John Coltrane played tenor sax.
\item  Thelonious Monk played gigs with John Coltrane.	\\
	John Coltrane played gigs with Thelonious Monk.
\item  All professional piano players have big hands.	\\
	Piano player Bud Powell had big hands.
\item  Bud Powell suffered from severe mental illness.	 \\
	All piano players suffer from severe mental illness.
\item John Coltrane was deeply religious.	 \\
John Coltrane viewed music as an expression of spirituality.
\end{compactlist}

\noindent
\problempart 
\label{pr.MartianGiraffes}
Consider the following sentences: 
\begin{compactlist}%[label=(\alph*)]
\item[G1] \label{itm:at_least_four}There are at least four giraffes at the wild animal park.
\item[G2] \label{itm:exactly_seven} There are exactly seven gorillas at the wild animal park.
\item[G3] \label{itm:not_more_than_two} There are not more than two Martians at the wild animal park.
\item[G4] \label{itm:martians} Every giraffe at the wild animal park is a Martian.
\end{compactlist}

Now consider each of the following collections of sentences. Which are jointly possible? Which are jointly impossible?
\begin{compactlist}
\item Sentences G2, G3, and G4
\hfill \myanswer{Jointly possible}
\item Sentences G1, G3, and G4
\hfill \myanswer{Jointly impossible}
\item Sentences G1, G2, and G4
\hfill \myanswer{Jointly possible}
\item Sentences G1, G2, and G3
\hfill \myanswer{Jointly possible}
\end{compactlist}

\problempart Consider the following sentences.
\begin{compactlist}%[label=(\alph*)]
\item[M1] \label{itm:allmortal} All people are mortal.
\item[M2] \label{itm:socperson} Socrates is a person.
\item[M3] \label{itm:socnotdie} Socrates will never die.
\item[M4] \label{itm:socmortal} Socrates is mortal.
\end{compactlist}
Which combinations of sentences are jointly possible? Mark each ``possible'' or ``impossible.''
\begin{compactlist}
\item Sentences M1, M2, and M3
\item Sentences M2, M3, and M4
\item Sentences M2 and M3
\item Sentences M1 and M4
\item Sentences M1, M2, M3, and M4
\end{compactlist}

\problempart
\label{pr.EnglishCombinations2}
Which of the following is possible? If it is possible, give an example. If it is not possible, explain why.
\begin{compactlist}
\item A valid argument that has one false premise and one true premise
\item[] \myanswer{Yes: `All whales are mammals (\emph{true}).  All mammals
  are plants (\emph{false}). So all whales are plants.' }
\item A valid argument that has a false conclusion
\item[] \myanswer{Yes. (See example from previous exercise.)}
\item A valid argument, the conclusion of which is a necessary falsehood
\item[] \myanswer{Yes: `$1+1=3$. So $1+2=4$.'}
\item An invalid argument, the conclusion of which is a necessary truth
\item[] \myanswer{No. If the conclusion is necessarily true, then there is no way to make it false, and hence no way to make it false whilst making all the premises true.} 
\item A necessary truth that is contingent
\item[] \myanswer{No. If a sentence is a necessary truth, it cannot
  possibly be false, but a contingent sentence can be false.} 
\item Two necessarily equivalent sentences, both of which are necessary truths
\item[] \myanswer{Yes: `4 is even', `4 is divisible by 2'.} 
\item Two necessarily equivalent sentences, one of which is a necessary truth and one of which is contingent
\item[] \myanswer{No.  A necessary truth cannot possibly be false,
  while a contingent sentence can be false.  So in any situation in
  which the contingent sentence is false, it will have a different
  truth value from the necessary truth. Thus they will not necessarily
  have the same truth value, and so will not be equivalent.}
\item Two necessarily equivalent sentences that together are jointly impossible
\item[] \myanswer{Yes: `$1+1=4$' and `$1+1=3$'.} 
\item A jointly possible collection of sentences that contains a necessary falsehood
\item[] \myanswer{No. If a sentence is necessarily false, there is no way to make it true, let alone it along with all the other sentences.}
\item A jointly impossible set of sentences that contains a necessary truth
\item[] \myanswer{Yes: `$1+1=4$' and `$1+1=2$'.}
\end{compactlist}

\problempart
Which of the following is possible? If it is possible, give an example. If it is not possible, explain why.

\begin{compactlist}
\item A valid argument, whose premises are all necessary truths, and whose conclusion is contingent
\item A valid argument with true premises and a false conclusion
\item A jointly possible collection of sentences that contains two sentences that are not necessarily equivalent
\item A jointly possible collection of sentences, all of which are contingent
\item A false necessary truth
\item A valid argument with false premises
\item A necessarily equivalent pair of sentences that are not jointly possible
\item A necessary truth that is also a necessary falsehood
\item A jointly possible collection of sentences that are all necessary falsehoods
\end{compactlist}
