%!TEX root = forallxyyc.tex

\chapter{Notes on accessibility}

Special attention has been paid to make this HTML version as
accessible as possible, especially to readers using Assistive
Technology (AT), such as screen readers. It has, however, not
been extensively tested. If you are using it with a screen reader or
Braille terminal, or are helping a student who relies on such tools
(e.g., as instructor or accessibility advisor), and you have feedback
please \href{mailto:rzach@ucalgary.ca}{get in touch}!

\paragraph{Navigation and styles} We have adopted the HTML style
provided by \href{https://vlmantova.github.io/bookml/}{BookML},
developed for the Leeds mathematics department by
\href{https://eps.leeds.ac.uk/maths/staff/4058/dr-vincenzo-l-mantova}{Vincenzo
Mantova}. It provides a collapsible navigation menu, buttons at the
top of each page to select the font size, switch between serif and
sans-serif font, and black-on-white, dark-on-sepia, or light-on-dark
display options.

\paragraph{Symbols and formulas} All logical symbols and formulas in
this book are converted to MathML and displayed using
\href{https://www.mathjax.org/}{MathJax}. MathJax provides additional
\href{https://docs.mathjax.org/en/latest/basic/accessibility.html}{accessibility
features} for formulas, which are found in the MathJax context
menu---activate/right click on any formula to activate it. If your
screen reader does not read out formulas such as $\forall x(A(x) \land
B(x))$ you may need to activate Speech Output in the Speech submenu of
the MathJax accessibility menu.

For reference, here is a table of all the symbols used in the text,
how they are (probably?) pronounced, and what they are called in the
text. In the rightmost column we provide a suggested way to enter them
using ASCII characters, if inserting special symbols (in a homework
assignment or email to your instructor, say) is not an option.

\begin{tabular}{lllll}
  \lxBeginTableHead{}Symbol\lxTableColumnHead{} & Pronounciation\lxTableColumnHead{} & Meaning\lxTableColumnHead{} & ASCII equivalent\lxTableColumnHead{}\\
  \hline\lxEndTableHead
  $\enot$ & not sign & logical not & \verb+~+ or \verb+-+\\
  $\eor$ & or & logical or & \verb+\/+\\
  $\eand$ & and & logical and & \verb+/\+ or \verb+&+\\
  $\eif$ & right arrow & conditional & \verb+->+ or \verb+>+\\
  $\eiff$ & left right arrow & biconditional & \verb+<->+ or \verb+<>+\\ 
  $\ered$ & up tack & contradiction & \verb+_|_+ or \verb+!?+\\
  $\forall$ & for all & universal quantifier & \verb|A| or \verb|@|\\
  $\exists$ & there exists & existential quantifier & \verb|E| or \verb|3|\\
  $\therefore$ & therefore & therefore & \verb|:.|\\
  $\proves$ & right tack & proves & \verb+|-+\\
  $\nproves$ & does not prove & does not prove & \verb+|/-+\\
  $\entails$ & true & entails & \verb+|=+\\
  $\nentails$ & not true & does not entail & \verb+|/=+\\
  $\ebox$ & white square & necessary & \verb+[]+\\
  $\ediamond$ & white diamond & possible & \verb+<>+
\end{tabular}

Subscripts should be pronounced by screen readers, although
if the subscript is a number, they may not be. They can be
represented by an underline, e.g., $A_2$ as \verb|A_2|.

The expressions `\blank{}' and `\ifeff' are used throughout the textbook.
`Iff' is short for `if and only if'. The HTML versions of both are
provided with ARIA labels to help screen readers pronounce them
properly (i.e., as `blank' and `if-eff').

\paragraph{Proofs} The natural deduction proofs in
\cref{ch.NDTFL,ch.NDFOL,ch.ML} use vertical lines to indicate where
subproofs start and end. Such vertical lines extend from the
assumption line of the subproof to its last line and are displayed
between the line numbers and the formulas in any given line. This
makes proofs a special challenge for users with low vision or complete
loss of vision. 

To make these proofs accessible in this HTML version, proofs are coded
as tables. Each table line has four columns: the line number, a
subproof level indicator, a formula, and a justification. The subproof
level indicator is a number recording how many nested subproofs the
current line is contained in. It is 0 if the line is not contained in
a subproof, 1 if it is in a subproof, 2 if it is in a subproof nested
within another subproof, and so on. When reading out a subproof level
indicator, screen readers should also announce if a subproof has just
been closed on the previous line, and when a new subproof starts. The
table header rows and subproof level indicators are hidden so that
proofs visually appear as in the printed text.

Here is an example of such a proof:
\begin{fitchproof}
\hypo{wxyz}{(W \eor X) \eor (Y \eor Z)}\PR
\hypo{xy}{X \eif Y}\PR
\hypo{nz}{\enot Z}\PR
\open
	\hypo{wx}{W \eor X}\AS
	\open
		\hypo{w}{W}\AS
		\have{wy1}{W \eor Y}\oi{w}
	\close
	\open
		\hypo{x}{X}\AS
		\have{y1}{Y}\ce{xy, x}
		\have{wy2}{W \eor Y}\oi{y1}
	\close
	\have{wy3}{W \eor Y}\oe{wx, w-wy1, x-wy2}
\close
\open
	\hypo{yz}{Y \eor Z}\AS
	\have{y}{Y}\ds{yz, nz}
	\have{wy}{W \eor Y}\oi{y}
\close
\have{wy4}{W \eor Y}\oe{wxyz, wx-wy3, yz-wy}
\end{fitchproof}
It has 14 lines, with 3 premises, 2 levels of subproof nesting, and
two pairs of adjacent subproofs. For instance, the subproof beginning
on line 5 ends at line 6, and line 7 starts another subproof. So the
subproof levels of lines 6 and 7 is the same, but lines 6 and 7 are in
different subproofs. If you cannot see the subproof lines, you have to
pay special attention to how the subproof level numbers change
\emph{and} when a formula is an assumption. A screen reader should
announce line 7 as ``7, close subproof, 2, open subproof, $X$,
AS.''